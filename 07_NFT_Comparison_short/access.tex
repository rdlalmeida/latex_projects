\documentclass{ieeeaccess}
\usepackage{cite}
\usepackage{amsmath,amssymb,amsfonts}
\usepackage{algorithmic}
\usepackage{graphicx}
\graphicspath{{Images/}}
\usepackage{textcomp}

% \usepackage[english]{babel}
\usepackage[utf8]{inputenc}
\usepackage[T1]{fontenc}
\usepackage{csquotes}
\usepackage{float}
\usepackage{enumerate}
\usepackage{lmodern}

\setlength{\marginparwidth}{2cm}
% \usepackage[disable]{todonotes}
\usepackage[showframe=false]{geometry}
\usepackage{amsthm}
\usepackage{subfiles}

\usepackage{siunitx}
\usepackage{multirow}
\usepackage{lscape}
\usepackage{booktabs}
\usepackage{tabularx} 
\usepackage[nottoc,numbib]{tocbibind}
\usepackage[super]{nth}

\usepackage[colorlinks=true, allcolors=blue]{hyperref}
\usepackage{xurl}

\usepackage{authblk}
\usepackage{changepage}

\usepackage[labelsep=period]{caption}
\usepackage[justification=centering]{caption}
\usepackage{soul}

\usepackage[romanian]{babel}
\usepackage{combelow}


% \captionsetup[table]{name=TABLE}
\renewcommand{\thetable}{\Roman{table}}

\hyphenation{Block-cha-in}
\hyphenation{block-cha-in}

% \newtheorem*{rqa1}{RQA1}
% \newtheorem*{rqa2}{RQA2}

% \newtheorem*{rqb1}{RQB1}
% \newtheorem*{rqb2}{RQB2}

% \newtheorem*{rqc1}{RQC1}
% \newtheorem*{rqc2}{RQC2}
% \newtheorem*{rqc3}{RQC3}
% \newtheorem*{rqc4}{RQC4}

\setlength\extrarowheight{2pt}

\def\BibTeX{{\rm B\kern-.05em{\sc i\kern-.025em b}\kern-.08em
    T\kern-.1667em\lower.7ex\hbox{E}\kern-.125emX}}
\begin{document}
% \history{Date of publication xxxx 00, 0000, date of current version xxxx 00, 0000.}
\doi{}

\title{Analysis of a Non-Fungible Token centric Blockchain Architecture and Comparison with a General Purpose Blockchain}
\author{\uppercase{Ricardo Lopes Almeida}\authorrefmark{1,2},
    % \IEEEmembership{Fellow, IEEE},
    Fabrizio Baiardi\authorrefmark{2},
    Damiano Di Francesco Maesa\authorrefmark{2} \IEEEmembership{Fellow, IEEE},
    % \IEEEmembership{Member, IEEE}
    Laura Ricci\authorrefmark{2}
}

\address[1]{Universit\`{a} di Camerino, 62032 MC, Camerino, Italia (e-mail: ricardo.almeida@unicam.it)}
\address[2]{Dipartimento di Informatica, Universit\`{a} di Pisa, 56127 PI, Pisa, Italia}

% \tfootnote{This work was supported in part by the U.S. Department of Commerce under Grant BS123456.}

% \markboth
% {Author \headeretal: Preparation of Papers for IEEE TRANSACTIONS and JOURNALS}
% {Author \headeretal: Preparation of Papers for IEEE TRANSACTIONS and JOURNALS}

\corresp{Corresponding author: Ricardo Lopes Almeida (e-mail: ricardo.almeida@unicam.it).}

\begin{abstract}
    Blockchain is one of the most disruptive technology in recent years. It has given a new meaning to decentralisation and through it brought the first true digital currency, i.e., cryptocurrency, a concept that has forced profound changes in the financial world. Cryptocurrencies are no longer amateur experiments and are more and more valid investments that compose larger and larger shares of investment portfolios. Cryptocurrencies work on top of the \textit{Fungible Token (FT)} concept, which can be programmatically adapted to change the nature of these token and transform them into \textit{Non Fungible Tokens (NFTs)}. NFTs are among the most recent and promising additions to the blockchain universe. While FTs are used to implement digital currencies, where these tokens emulate coins and bank notes used in normal, fiat currencies, NFTs are used to establish ownership of objects, digital and otherwise. The potential of this new addition is still being explored. Unlike FTs, their non-fungible counterparts are used to store data in the blockchain. This is how they establish the ownership relations and how the digital or physical object whose ownership is established by the token are identifies. This fundamental nature in operation provide NFTs with a larger landscape of possible implementation architectures than the fungible version. This paper focuses on the architecture of Flow, the first blockchain created with NFT support as its central tenet, and it works and how it compares to Ethereum, the first blockchain to provide support to NFT-based projects, but that still operates as a general purpose blockchain.
\end{abstract}

\begin{keywords}
    Blockchain, Non-Fungible Tokens, Ethereum, Flow, Solidity, Cadence
\end{keywords}

\titlepgskip=-21pt

\maketitle

% SUBFILES
\subfile{./Sections/01_Introduction.tex}

\subfile{./Sections/02_RelatedWorks.tex}

\subfile{./Sections/03_FlowBlockchain.tex}

\subfile{./Sections/04_ArchitectureComparison.tex}

\subfile{./Sections/05_Conclusion.tex}

% SUBFILES
\twocolumn

\bibliographystyle{IEEEtran}
\bibliography{Bibliography.bib}

\EOD
\end{document}
