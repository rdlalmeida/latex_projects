\documentclass[../access.tex]{subfiles}
\graphicspath{{\subfix{../Images}}}

\begin{document}
\section{Introduction}
\label{sec:introduction}
\textit{Non Fungible Tokens (NFTs)} are one of the latest major features introduced in the decentralised blockchain universe. NFTs follow the tokenisation trend introduced by Bitcoin \cite{Nakamoto2008}, the first example of a public blockchain and cryptocurrency, and establish an almost antagonist concept to the latter. Cryptocurrencies are established in blockchains with \textit{Fungible Tokens (FT)}, a concept akin to coins and notes used with fiat currencies. As such, FTs are interchangeable and can be divided in subunits without loss of value, just like regular money. NFTs on the other hand, as the name implies, cannot be subdivided or exchanged by another token with similar value. A good analogy for NFTs are works of art, such as painting and sculptures, that is, unique objects valued individually, irregardless if all were produced by the same artist, just like each NFT has its own characteristics and value independent of the contract that issued it.
\par
NFTs are implemented through smart contracts, which are software scripts that can be deployed, i.e., stored, in a blockchain block and executed distributively by the active nodes in the network. In a smart contract enabled blockchain, smart contract instructions are sent to execute in a subset of active nodes and their results are then confirmed and validated by the rest of the network towards achieving a computational consensus. The aggregated resources of a blockchain network, e.g., computational power, memory, storage, etc. are often referred as a \textit{Blockchain Virtual Machine}. Most modern blockchains implement Turing-complete virtual machines used to run smart contracts deterministically and implement non-fungible tokens.
\par
The first example of an NFT was introduced by a pair of digital artists through the Quantum project \cite{Exmundo2023}. They used Namecoin, a blockchain forked from Bitcoin, but with the addition of a \textit{blockchain transaction database} that could be used to store non-transactional data \cite{Loibl2014}. This allowed the artists to store a digital produced image directly in the blockchain and use the transactional data to establish a unique ownership relation between the record and an account address, thus creating the first real world instance of a \textit{Non-Fungible Token}. The NFT concept was known at the time and, as such, this project was soon followed by other, more technologically sound, projects. The majority took advantage of Ethereum and the smart contract support introduced when this blockchain, which promised a easier method to create NFTs than the one used in Quantum. It did not took much for Ethereum to become the most popular public blockchain for NFT-based projects. Currently, the top 10 NFT projects ranked by market cap are all Ethereum based \cite{CoinGecko2024}.
\par
Event though Ethereum and smart contracts added significant programmatic flexibility to the universe of blockchain solutions, it is still a notoriously difficult network to scale and it tries to establish a rate of one block per 12 ~ 15 seconds, which can limit significantly the throughput of the chain, as well as the scope of applications supported. One of such applications was the \textit{CryptoKitties} project launched back in 2017 \cite{Dapper2017}, one of the earliest NFT contracts deployed in Ethereum. This project was created by \textit{Dapper Labs} (known previously as \textit{Axiom Labs}) and extended the NFT concept with a new usability layer that was absent from other similar projects. The contract minted a \textit{CryptoKitty}, a NFT representing a digital cat-like creature, and each kitty token was characterised by a unique genome parameter, an internal 256-byte string from which the visual characteristics of the token were derived from. Parameters as eye color, skin color, ear type, etc were encoded in portions of the genome string. The innovative aspect of this project was that two CryptoKitties could be "bred" to generate a new one with a genome string that derived from the parent's genome. Dapper Labs established the genome mechanics such that new traits were acquired somewhat randomly (pure randomness is hard to achieve in a purely deterministic blockchain environment) and some traits were rare and hard to obtain than others. This new approach translated into a peak of popularity and a surge in Ethereum transactions, which were used to acquire new kitties and breed new pairs, and that end up exposing the scalability limitations of Ethereum \cite{bbc2017}.
\par
Dapper Labs initially tried to solve these issues from within the Ethereum blockchain, but at some point it became clear that the blockchain needed architectural modifications to be able to overcome these throughput limitations. As such, instead of trying to "fix" Ethereum, Dapper Labs launched Flow in 2020 instead \cite{Gharegozlou2019}, a new blockchain solution developed from scratch centered around supporting NFTs and related mechanics. Flow presents several key differences from Ethereum, from how nodes behave in the network, the consensus algorithm used, to how data is stored and accessed in the chain, as well as similarities, such as defining and using a native cryptocurrency token to regulate blockchain operations (\textit{gas} in Ethereum) and smart contract support. Yet, Flow claims to present the same level of NFT functionalities as Ethereum and other similar NFT ready blockchains, but approaching the concept from a fundamental different point.
\par
This article presents an introduction to the Flow blockchain and how it organises itself from the architecture standpoint, followed by the introduction and analysis of a pair of simple implementations of a NFT minter smart contract, one in Cadence, Flow's smart contract programming language, and another in Solidity, Ethereum's equivalent. We finish this publication by comparing both architectures and implementations towards determining the merits of Flow's claims be a viable and optimised alternative from NFT-based projects.
\par
The rest of this article is structured as follows: Section \ref{sec:related_works} provides an overview of publications relevant to this work. In Section \ref{sec:flow_blockchain} we provide an introduction to Flow, the blockchain central to this exercise. Section \ref{sec:architecture_comparison} provides a comparison between implementations of NFT smart contracts in both Cadence and Solidity, as well as the supporting blockchain architectures, namely Flow and Ethereum. This article concludes with Section \ref{sec:conclusion}.
\end{document}