\documentclass[10pt]{article}

\usepackage{amssymb}
\usepackage{graphicx}
\graphicspath{{Images/}}
\usepackage{subfiles}
\usepackage[english]{babel}
\usepackage[utf8]{inputenc}
\usepackage[T1]{fontenc}
\usepackage{csquotes}
\usepackage{float}
\usepackage{enumerate}
\usepackage{lmodern}
\usepackage[
    backend=biber,
    bibencoding=latin1
]{biblatex}
\usepackage{todonotes}

\usepackage{booktabs}
\usepackage{multirow}
\usepackage{tabularx}
\usepackage{chngpage}
\usepackage[colorlinks=true, allcolors=blue]{hyperref}
\usepackage[nottoc,numbib]{tocbibind}

\usepackage{listings}
\usepackage{color}

\definecolor{dkgreen}{rgb}{0, 0.6, 0}
\definecolor{gray}{rgb}{0.5, 0.5, 0.5}
\definecolor{mauve}{rgb}{0.5, 0, 0.82}

\lstset{
    frame=tb,
    language=Java,
    aboveskip=3mm,
    belowskip=3mm,
    showstringspaces=false,
    columns=flexible,
    basicstyle={\small\ttfamily},
    numbers=none,
    numberstyle=\tiny\color{gray},
    keywordstyle=\color{blue},
    commentstyle=\color{dkgreen},
    stringstyle=\color{mauve},
    breaklines=true,
    breakatwhitespace=true,
    tabsize=3
}

\hypersetup{
    pdftitle={Proposal Article},
    pdfauthor={Ricardo Almeida},
    pdfsubject={Non Fungible Token Implementation},
    pdfkeywords={blockchain, non fungible tokens, cadence, flow, ethereum, solidity},
    bookmarksnumbered=true,
    bookmarksopen=true,
    bookmarksopenlevel=1,
    colorlinks=true,
    pdfstartview=Fit,
    pdfpagemode=UseOutlines,
    pdfpagelayout=SinglePage,
    linkcolor=black
}

\hyphenation{block-chain}
\hyphenation{Ether-eum}
\hyphenation{ether-eum}
\hyphenation{break-throughs}

\usepackage{authblk}
\bibliography{Bibliography.bib}

\author[1]{Ricardo Lopes Almeida}
\author[2]{Fabrizio Baiardi}
\author[3]{Damiano Di Francesco Maesa}
\author[4]{Laura Ricci}

\affil[1, 2, 3, 4]{Dipartimento di Informatica, Università di Pisa, Italia}
\affil[1]{Università di Camerino, Italia}

\title{Contract-based Storage versus Account-based Storage: a Comparison among Implementation Strategies for Non-Fungible Tokens }

\begin{document}

\maketitle

\begin{abstract}
    Non-Fungible Tokens (NFTs) are a recent and promising features from blockchain technology. NFTs provide a novel method to establish ownership of unique digital objects. Alongside with cryptocurrencies, these provide a new technological landscape that can revolutionise economies by providing an efficient and transparent method to establish ownership and transact value between peers.
    \par
    Ethereum was the first blockchain to support NFT-based projects, but this development ecosystem is still growing. The popularity of NFTs encouraged other blockchains to extended their support to include these tokens, but others were built from scratch using the NFT as the cornerstone. One of these blockchains is Flow, a Proof-of-Stake blockchain founded in 2020 by the same team that deployed one of the earliest NFT projects in Ethereum. Flow was presented as a solution to the development problems and limitations identifies by the creator team in their Ethereum experience.
    \par
    This paper presents the results of an objective comparison between two implementations of basic NFT smart contract into two architecturally different blockchain, namely Ethereum and Flow. The two implementations are compared based on common operations and features, operational costs and ease of implementation, with the goal of determining if Flow, a specialised blockchain, offers significant advantages compared to the generalised approach provided by Ethereum.
\end{abstract}


%% \linenumbers

%% main text
\subfile{./Sections/01_Introduction.tex}

\subfile{./Sections/02_Related_Works.tex}

\subfile{./Sections/03_Background.tex}

\subfile{./Sections/04_NonFungibleTokens.tex}

\subfile{./Sections/05_Solidity-Ethereum_NFT_implementation.tex}

\subfile{./Sections/06_Cadence-Flow_NFT_implementation.tex}

\subfile{./Sections/07_Results_and_Comparison.tex}

\subfile{./Sections/08_Conclusion.tex}

\printbibliography

\end{document}