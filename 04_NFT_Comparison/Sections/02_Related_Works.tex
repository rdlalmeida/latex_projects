\documentclass[../main.tex]{subfiles}
\graphicspath{{\subfix{../Images}}}

\begin{document}
\section{Related Works}
\label{sec:related_works}
Blockchain based academic research is a relatively recent event. When narrowed down to only involving NFTs, the universe of relevant publication diminishes significantly. Nonetheless, there are some relevant publications with NFTs as the centre topic. In \cite{Chiacchio2022} the authors propose an blockchain-based NFT system to track and trace pharmaceutical products. They present a solution based in the \textit{VeChain Thor} blockchain, a public, Proof-of-Authority blockchain where they use NFTs to abstract pharmaceutical items in transit. When the item arrives to a checkpoint, an operator uses a QR reader to read a code in the product that automatically updates the NFT lifecycle in the blockchain, thus implementing a \textit{digital twin} approach to the problem. The same NFT-based tracing concept was also explored in \cite{Bal2019}, though with a more general approach than the previous authors. For this case, the researchers used Hyperledger, a blockchain solution produced by the Linux Foundation to design custom private blockchain systems with smart contract and NFT support. Hyperledger is also used in \cite{Karandikar2021}, where the authors developed a private blockchain where they modeled an environment where microgeneration of electricity occurs. They used NFTs and FTs (Fungible Tokens) to abstract the actors and values respectively. Actors such as solar panels, electric vehicles, batteries, users, utility companies, etc, were abstracted as NFTs, while values, e.g., money and electric energy, are represented with cryptocurrencies. FTs flow in and out of NFTs in this system emulating how electricity is produced, consumed and sold among the actors.
\par
The tokenisation of real world items appears to be a popular approach. \cite{Regner2019} presents another NFT-based blockchain solution to manage event tickets. The unique nature of NFTs within a blockchain is particularly useful to represent real objects that require the same characteristics. An event ticket is a finite quantity by default and these are ofter individualised with serial numbers, seat allocations, etc., which makes it a perfect candidate for NFT abstraction. The ticket characteristics are trivial to encode in the NFT metadata, thus the functionality of it is only limited to the ability of operators in the venue to access the blockchain. The authors presented a prototype developed in the Ethereum network, a popular, albeit expensive and hard to scale, solution. \cite{Wadhwa2024} follows a similar theme and presents a NFT-based e-voting system. In this proposal, the author uses a type of non-transferable NFTs called \textit{Soulbound NFTs} to abstract the voters, a very similar approach used in \cite{Sagar2023}.
\par
The proposals indicated thus far do explore NFTs as critical elements of a solution, but they do not go into great detail regarding implementation details or explore alternative architectures. The authors in \cite{Hong2019} go briefly into the details of an NFT implementation in an Hyperledger environment towards providing an extensive model to create additional token, but following the token standards defined in Ethereum, namely \textit{ERC-721}. An interesting study on NFT architecture was presented in \cite{Yang2022} where the author present a generic NFT architecture along side with a series of software connectors intended to achieve interoperability among different blockchains, but they do not go into implementation details. They do present results from an experiment implemented in Ethereum, but without sufficient details to replicate the exercise.
\par
In all of the cases considered, all emphasised the use of existing NFT standards but their mentions were limited the to the ERC standards from the Ethereum network, which is understandable given that most projects referenced in this articles are based in Solidity smart contracts deployed in the Ethereum network.Though some mention the existence of other blockchains with NFT capabilities, only \cite{Wang2021}, \cite{Razi2024}, and \cite{Guidi2023} mention the Flow blockchain specifically, with \cite{Guidi2023} providing the most detailed introduction.

\subsection{Our Contribution}
The proposals considered present some specifics about the implementation of NFTs in each cases, but none approaches the data storage approach for these tokens. This is an instrumental aspect when using these tokens, given how they were used to transport and secure data in the solutions considered. The concrete implementations indicated used either the Ethereum blockchain or Hyperledger, which produces custom private blockchains. These blockchains support NFTs derived from their smart contract capabilities, but since Hyperledger used the token standards from Ethereum instead of their own, the comparison between these solutions is not ideal. Our proposal specifies two distinct blockchains with different architectures, each implementing their own token standards to govern NFT creation and activity, but promising similar functionalities. As far as this writing, no other article to our knowledge was published centered on NFT storage mechanics nor considered other blockchain architectures with sufficient detail to enact a fair comparison. As such, we propose the following contributions to this research field:
\begin{enumerate}
    \item{Compared to other publications in this area, we provide an introduction to the architectural aspects used to implement a smart contract capable of minting and operating with NFTs, with emphasis on the storage mechanism and control access structures used.}
    \item {We also provide an introduction to a less popular but more application specific blockchain towards establishing a reference point for a comparison.}
    \item {The paper concludes with an objective comparison between the architectures considered, as well as an evaluation over which has a better performance in an NFT-based application}
\end{enumerate}
\end{document}