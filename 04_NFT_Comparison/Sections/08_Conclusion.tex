\documentclass[../main.tex]{subfiles}
\graphicspath{{subfix{../Images}}}

\begin{document}
\section{Conclusion}
\label{sec:conclusion}
This paper presented a comparison between two blockchain architectures that implement the same Non-Fungible Token concept in fundamentally different ways, but achieving similar functionality. Ethereum was used as the reference blockchain for this purpose, given its popularity in academic research, extensive documentation and mature ecosystem, thus serving as a general purpose blockchain for this matter. Oppose to it was Flow, a much younger blockchain that was developed specifically to address the limitations that a general purpose blockchain, such as Ethereum, may present when dealing with NFT mechanics.
\par
This analysis suggests that, where NFTs are concerned, Flow offers a better, more specialised architecture to implement projects based in this concept. The account-based storage model used in Flow moves into a more decentralised approach to the whole framework as compared with the contract-based approach from Ethereum. NFTs in Flow are digital resources independent from the contract that mints them.
\par
Cost wise, the differences are more apparent. There is some irony in the fact that currently, the popularity of Ethereum is making the price of its native cryptocurrency, ETH, quite volatile and even more expensive when compared with other public blockchains with similar functionalities. ETH price is currently more influenced by external speculative events than moved by the utility of the token in the application ecosystem. When compared to FLOW, this difference is extreme. If price is a constrain in the project, Ethereum becomes quite expensive quickly.
\par
This difference was expectable, given that Flow justifies its existence to the difficulties encountered by its creators when deploying one of Ethereum's first NFT projects. Overall, Flow offers better performance and it is easier to operate than Ethereum since the blockchain was created around the concept of resource, which are perfect abstractions for NFTs. Flow achieves this at a considerable increase in complexity, both in Cadence, the language used to write smart contracts in Flow, and in the organisation of the network itself. Given that Ethereum has a more general purpose appeal, it would be interesting to compare how Flow could fare regarding non-NFT applications. As far as we could determine, Flow's virtual machine is as capable as Ethereum's,  but without proper experimental results, we need to limit Flow's superiority to NFT applications.
\end{document}