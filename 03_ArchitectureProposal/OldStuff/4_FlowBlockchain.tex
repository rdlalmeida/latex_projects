The Flow blockchain was formally launched in October 2020 using the popular \textit{Initial Coin Offering (ICO)}. Flow followed a trend up to that point, and established itself with the more efficient and energy friendly \textit{Proof-of-Stake (PoS)} consensus mechanism, as opposed to the \textit{Proof-of-Work (PoW)}, the norm in earlier blockchains \cite{Hentschel2019a}.
\par
In a Proof-of-Stake based blockchain, new blocks are "mined" by active nodes based on their \textit{stake} on the network, namely, the volume of the blockchain's native cryptocurrency the node currently holds in its account. Higher stakes mean higher probability of being awarded a block publication by the protocol that regulates the network, with the token incentives that such entitles, and vice versa. This process is more energy efficient and faster than the "classical" PoW consensus, where nodes pointlessly spend computer cycles solving cryptographic puzzles only to win a race towards its completion. Given the significant energy waste due to the popularity of PoW, a blockchain that stays clear of such nefarious protocol is not just a smart choice but a conscientious one as well \cite{Hentschel2019b}.
\par
Flow was created by the same team that created \textit{CryptoKitties}, one of the first NFTs based blockchain games in the Ethereum network to adopt the \textit{ERC-721} standard for Non-Fungible Tokens. Even though Ethereum was the first one to offer NFT support, it did so on top of a chain that did not envision such applications at the time of its conception. This was evident by the lack of flexibility, inherent security flaws in the code that implemented the tokens and its mechanics, as well as a general complexity in writing code and operating with these constructs in the Ethereum network. Ethereum was so ill prepared to deal with the popularity of this initiative that its network went down briefly due to its inability to cope with the added network traffic \cite{bbc2017}. Flow was developed with NFTs support as its main feature, prioritizing NFT creation and mechanics over other applications, a clear contrast over blockchains more focused on cryptocurrency transactions \cite{Gharegozlou2019}.
\par
Flow establishes a new computational paradigm in this context, a \textit{Resource Oriented Paradigm (ROP)}. It does so through \textit{Cadence}, a specialized programming language developed for Flow. \textit{Cadence} establishes \textit{ROP} through a special type of object, named \textit{Resource}. From a technological point of view, a Resource is akin to a structure, offering a significant degree of freedom on the type of data that it can encode. But Cadence regulates operations through the notion that every Resource is unique in Flow. This means that Resources need to be accounted at all times, and they cannot be copied, only moved. They can be stored in decentralized storage mediums and be loaded from them but, at all times, the Resource is either in storage or out of storage, never in the same state at the same time, something common and trivial with non-unique digital resources. These Resources can be created only through Smart Contracts and in Flow, and they operate similarly in other blockchains that support them.
\subsubsection{Account-based Storage in Flow}
Flow was conceived to be a NFT-oriented blockchain, unlike the usual, cryptocurrency focus of other alternatives. As such, data storage was emphasized regarding other blockchain characteristics. Other blockchains, Ethereum being a prime example, centralize NFT storage in the respective contract, i.e., the digital object - owning account key-value pair is saved related to the contract. This means that, if the contract is deleted, ownership of its issued NFTs disappears with it, as well as the contract-based NFTs.
\par
This mechanic was substantially changed in Flow. Storage in this blockchain is account-based. When a user creates an account in Flow, which is quite similar to creating a cryptocurrency wallet in any other blockchain in the sense that it is also based in the creation of an asymmetric encryption key pair, where the account address is derived from the public key and account access requires control of the corresponding private key. Storage also follows a key-value scheme, as it is common in this context, but every key stored is associated with an account by having its address, which is unique for every user in the ecosystem, appended to it. Storage space depends on the amount of Flow token currently deposited in the account, according to a rate of 100MB of storage per token \cite{flow2020b}.
\par
Each account can store data in either a \textit{public} or \textit{private} sub path. Public stored data can be read by any other account without restrictions, but any modifications require proof of ownership of the account by providing a signed transaction, i.e., the data reads occurs by submitting a digitally signed Cadence script with the required instructions for the act, which in the Flow context is referred as \textit{transaction}. Only the account owner can provide the required signature, since access to data in the private storage - for read or write purposes - always requires a signed transaction as well \cite{Hentschel2019a}.

\subsubsection{The case for Flow}
Flow strengthens its case as a prime candidate for a NFT-based project by offering a relatively high block rate and low gas fees, which translates into fast transactions and low overheads. It also uses a novel way to store data, using an account-based storage instead of the "traditional" contract-based storage in Ethereum and similar blockchains. The mechanism of storing resources under the user's account, it is uniquely apt to a context where sensible data needs to be transferred through the blockchain. Securing this data under an individual account, and set it private so that only the account owner can access it, is a feature that distinguishes this blockchain from similar ones. It is easy to understand the security problems behind a more "conventional" blockchain storage system, where all data is referenced from a single point of access, typically the regulating contract address, a centralizing element in a purely decentralized application.
\par
The reasons listed above single out Flow as an optimal choice for NFT-based development.