\documentclass[../main.tex]{subfiles}
\graphicspath{{\subfix{../Images}}}

\begin{document}
    Modern democracies are critical to technological advances, as they provide citizens with levels of comfort and security that allow them to divert their time and energy away from basic survival towards research and innovation. This is a transversal effect across society, with all sorts of mundane activities getting progressively easier and even automated, with each technological advance.
    Yet, the exercise that keeps these same modern democracies running smoothly still resists innovation. Most elements that characterize a modern society - banking, farming, food processing, education, healthcare, etc - have improved thanks to technology. All except voting. Except for a few minor upgrades on the voting act itself, the fundamental mechanics remain almost unchanged.
    \par
    From dropping clay shards in a pot, to selecting an option in a touchscreen, there has not been a change significant enough to bring this process to the same levels of practicality of other activities. Votes submitted in an electronic medium can be counted faster, but other than that, current electronic voting still limits voters geographically and does not contribute to make elections easier or cheaper to organize.
    \par
    This is not an indication of a lack of interest in the research community. In fact, the modernization of voting systems has been an active area of research in general and the digitalization of voting systems has been researched actively since the advent of commercial cryptography. A landmark article by Diffie and Hellman in 1976 \cite{Diffie1976} established the basis for commercial cryptography, a research area that was confined to governmental (i.e., military) applications until then. This publication spearheaded the development of several cryptographic tools, particularly schemes for symmetrical and asymmetrical encryption systems, as well as cryptosystems that were made available for the public based on diverse approaches to intractable problems that are trivial to solve in one direction but computationally unfeasible in the opposite. The \textit{Rivest-Shamir-Adleman (RSA)} \cite{Rivest1983} cryptosystem, which relies on the factorization of large prime numbers, and the \textit{ElGamal} \cite{ElGamal1984} cryptosystem, which uses the intractability of computing discrete algorithms to establish cryptographic operations, are among the most popular ones.
    \par
    In the specific context of e-voting, these advancements were followed by approximately three decades years of e-voting research based on a centralized, server-client paradigm that produced a significant number of academic proposals that used cryptographic tools such as encryption, digital signatures, one-way hash-function, etc. to create secure and transparent e-voting systems.
    \par
    This was the state of e-voting research until 2009. At the end of that year, blockchain broke into mainstream through Bitcoin, the first cryptocurrency, introduced through a landmark paper \cite{Nakamoto2008}. After Bitcoin's solidification as the first truly digital currency, others followed, improving not only the currency feature, but adding support to different applications. Six years later, the Ethereum blockchain \cite{Dannen2016} was launched and with it the concept of blockchain virtual machine, a functional abstraction of the cumulative resources available in the network nodes, which could be used to run code synchronously - referred to as Smart Contracts - in a subset of the network nodes. The virtual machine abstracted by the Ethereum network was named \textit{Ethereum Virtual Machine (EVM)} \cite{Antonopoulos2018}. Though Bitcoin already offered some capacity to execute scripted code in a distributed manner, this was not its main feature and thus was quite limited in that regard, Ethereum was the first blockchain to offer explicit support for Smart Contracts and since more and more blockchains have been launched with similar capabilities. As a consequence, Smart Contracts became almost ubiquitous in the blockchain ecosystem. Other examples of blockchains with smart contract support created after Ethereum include \textit{Cardano}, \textit{Solana}, \textit{Tron}, \textit{EOS}, \textit{Polkadot} and \textit{Flow} \cite{cointree2022}. 
    \par
    The applicability of blockchain to the e-voting context was clear from early on, and e-voting research migrate to this new approach to take advantage of these features. The first academic articles detailing e-voting systems using a blockchain appeared in 2016 and, since then, the interest in this new research approach has increased steadily. Blockchain is still a technology under development, and new features based on this concept have been introduced at a fast pace in recent years.
    \par
    This article presents a fully mobile, blockchain based electronic voting system that explores the concept of Non-Fungible Tokens (NFTs), one of the newest features put forward by the Smart Contract enabled Ethereum network. We discuss this concept in greater detail in Section \ref{introduction_nfts}.
    \par
    The rest of this paper is organized as follows: a state of the art regarding the evolution of e-voting systems is presented in section \ref{stateoftheart}. Section \ref{introduction_nfts} introduces the central element in the solution presented, mainly from a technical point of view. Section \ref{general_approach} begins by detailing a generalist approach to the problem at hand, followed with a description of the two approaches considered. A security analysis of the proposed solution follows with Section \ref{features_and_security}. The paper concludes with section \ref{conclusion} where the main conclusions and open problems are discussed.
\end{document}