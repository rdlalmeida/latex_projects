\documentclass[../main.tex]{subfiles}
\graphicspath{{\subfix{../Images}}}

\begin{document}
The history of e-voting systems research details an evolution process that was triggered by the first proposals resulting from the commercialisation of cryptography, up to the latest decentralised proposals using blockchain as the technological basis.
\par
Cryptography was mostly used for military purposes throughout history, but its commercialisation has opened up new avenues for secure communication and data integrity in various fields, including voting. This shift has enabled the development of innovative e-voting systems that aim to enhance transparency and security in electoral processes, where it was used to protect communications on the battlefield. Commercial cryptographic applications were inexistent until the publication of \cite{Diffie1976} in 1976. This paper showed how to create an encryption cryptosystem from an intractable mathematical problem. Problems such as the factorisation of large primes and prime-based logarithms are practically impossible to solve in one direction, but given a solution, it is trivial to verify its correctness. For example, for a given large prime number, it is very time- and effort-consuming to find two smaller prime numbers that, when multiplied, result in that large prime. On the other hand, it is trivial to verify if any given pair of smaller prime numbers is a solution, i.e., we get the large prime when the smaller primes are multiplied by each other.
\par
\cite{Diffie1976} presented the first proposal for a non-military, non-proprietary cryptosystem, but it was quickly followed by other articles detailing alternative solutions and exploring other sources for intractable problems. \cite{Shamir1979}, \cite{Chaum1981}, \cite{ElGamal1984}, and \cite{Rivest1983} proposed the first commercial cryptosystems based on symmetrical and asymmetrical encryption keys. The applicability of these cryptographic tools in an e-voting context was clear, and the first e-voting proposals based on these concepts soon followed. These early systems are characterised by cryptographic tools derived from research on commercial cryptography, such as blind signatures \cite{Chaum1983}, mix-nets \cite{Chaum1988}, homomorphism in threshold cryptosystems \cite{Shamir1979}, and cryptographic proofs \cite{Goldwasser1986}.
\par
Research in e-voting systems progressed towards the establishment of security classification criteria, such as \textit{accuracy}, \textit{privacy}, \textit{eligibility}, \textit{verifiability}, \textit{convenience}, \textit{flexibility}, \textit{mobility}, and \textit{robustness}. A simple example that illustrates this process is the usage of asymmetrical encryption keys to encrypt voter data, thus protecting the \textit{privacy} of the voter. A proposal that uses such a scheme can claim that it establishes voter \textit{privacy}. Yet, a formal definition of such criteria has been notoriously absent from related literature. \cite{Neumann1993} was among the first to attempt such characterisation, with subsequent publications, such as \cite{Fujioka1992}, \cite{Baraani1995}, \cite{Juang1997}, \cite{Ku1999}, \cite{Lee2000}, \cite{Joaquim2003}, \cite{Baiardi2005}, and \cite{Chaum2007}, continuing this trend. These articles followed the rationale that the more secure a system is, i.e., the more security criteria it implements, the more it can assure a user that he/she can trust his/her choice to it. Over time, these cryptographic tools became a fixture in all e-voting proposals in this initial 30-year window of research in centralised e-voting systems. This is a limiting paradigm since it constrains the whole system by establishing a single point of attack or failure while also reducing system scalability, due in great part to the demand of a large amount of resources, such as primary and secondary storage, computational power, network bandwidth, etc., to implement these criteria.
\par
Scalability is an important characteristic that can hinder a wide adoption of the proposed system. Contemporary elections can go from simple exercises, where the system is expected to process up to a thousand votes at one point, to nationwide events that require the processing of millions of votes instead. The relationship between the scalability of a system and the amount of available resources is quite evident in the analysed literature. Proposals from this era confirm that the ones that satisfy the most security criteria also establish a computationally complex and demanding system that is often limited to small-scale elections. As an example, in \cite{Cranor2002}, \cite{Nurmi1991}, \cite{Iversen1992}, and \cite{Niemi1999}, this trade-off was shifted towards security and transparency at the expense of scalability; hence, these systems are all limited to small-scale elections, as the authors declare. Conversely, \cite{Benaloh1994}, \cite{Benaloh1986a}, \cite{Park1994}, \cite{Juang2002}, \cite{Okamoto1996}, \cite{Okamoto1998}, \cite{Magkos2001}, and \cite{Moran2006} present simpler scalable systems, but whose adoption in large-scale elections requires a sacrifice in security and transparency due to a lower number of security criteria implemented.
\par
We were able to find examples of real-world implementation of e-voting systems. For this case, we were only interested in systems that addressed the criterion of \textit{mobility}, i.e., a voting system that does not restrict voters geographically, in part because these proposals did not follow any of the academic ones that preceded them. There was no interest in electronic proposals that limited their users to traditional polling places, since they infuse a degree of privacy and security that derives solely from the surrounding election apparatus. The more notable exercises in recent history that match this criteria were run in Canada in 2013 \cite{Goodman2014}, Estonia in 2005 \cite{Heiberg2005}, Switzerland in 2005 \cite{Binder2019}, Norway in 2011 \cite{Barrat2012}, France in 2012 \cite{Pinault2012}, and Australia in 2015 \cite{Halderman2015}.
\par
This analysis uncovered a separation between academic e-voting research and the systems that are experimented with in real-world scenarios, given that most real-world remote solutions considered were developed through contracts awarded to private companies. As a consequence, specific technical details are kept private as patented intellectual property, with any technical information about them available only from review reports, which makes any further analysis a difficult endeavour. In large-scale elections, the scalability effort quickly grows out of the capacity of small organisations and individuals, leaving governments as the only organisations with enough resources to implement them. This generation of e-voting systems proved that the concept is sound and that it is possible to use electronic means to establish elections with greater accuracy and security for their participants.
\par
A shift towards a decentralised approach to this issue initiated with proposals that used blockchain merely for transport and record, in part as a consequence of the "imposition" of using Bitcoin's blockchain, since it was the only mature public blockchain available at the time. Proposals such as \cite{Zhao2016}, \cite{Cruz2016}, \cite{Bistarelli2017}, \cite{Lee2017}, \cite{Shaheen2017}, \cite{Wu2017}, \cite{Dimitriou2020}, and \cite{Bartolucci2018} used a script function available in Bitcoin transactions to add voting information to the blockchain data. These authors used Bitcoin's \textit{OP\_RETURN} function, which receives an 83-byte wide string as input and adds it to the transaction metadata as the function's output (depends on the version of the Bitcoin protocol). Furthermore, this method is infeasible for large-scale use because a Bitcoin transaction necessarily involves exchanges worth a considerable amount of money, as well as being a notoriously hard-to-scale blockchain due to its high block rate. Bitcoin adds a new block every 10 minutes, which severely limits the rate of operations that this blockchain can withstand.
\par
The popularisation of public blockchains attracted interest from other platforms, which triggered the development of software frameworks used to create and deploy custom blockchains with proprietary access control and offered more flexibility to applications. As such, researchers deployed custom-made solutions in customisable frameworks such as Hyperledger Fabric, Quorum, and Multichain. Publications such as \cite{Kirby2016}, \cite{BenAyed2017}, \cite{Chaieb2018}, \cite{Burhanuddin2018}, \cite{Zhang2018}, \cite{Khan2018}, \cite{Murtaza2019}, \cite{Faour2019}, \cite{Killer2020}, \cite{Han2020}, \cite{Vivek2020}, \cite{Mani2022}, \cite{Zhou2020}, \cite{Alvi2022}, \cite{Hassan2022}, \cite{Vidwans2022}, and \cite{Matile2019} take advantage of private blockchains tailored to a voting application. As such, they paid for the increased flexibility with a lack of network support. It is difficult to establish a privately accessible network with enough active nodes that can establish a satisfactory level of redundancy.
\par
A significant breakthrough arrived with the introduction of the smart contract through the Ethereum blockchain, specifically through the implementation of a \textit{Touring-complete} processing platform, named \textit{Ethereum Virtual Machine (EVM)}, that can execute code scripts in a decentralised fashion by splitting and distributing the instructions through the active nodes in the network. This opened up new research avenues in the e-voting environment. A few years after Ethereum's debut, the first proposals used smart contracts to establish the logic of the system in a decentralised approach, such as \cite{McCorry2017}, \cite{Koc2018}, \cite{Dagher2018}, \cite{Fusco2018}, \cite{Hjalmarsson2018}, and \cite{Mols2020}.
\par
Some researchers did base their solution on the Ethereum network, but they did not resort to smart contracts; instead, they explored the increased flexibility and additional application support of this network, employing similar methods as the ones in earlier proposals. \cite{Hardwick2018}, \cite{Wang2018}, \cite{Hsiao2018}, \cite{Lai2018}, \cite{Shukla2018}, and \cite{Vo-Cao-Thuy2019} provide examples of this strategy.
\par
The same time period saw several blockchain-based proposals for electronic voting systems in a real-world scenario. Small organisations are putting forward these proposals. But unlike the centralised approach, these solutions have significant overlap with the academic proposals considered. Among these real-world examples, we cite \textit{Follow My Vote} \cite{Ernest2021}, \textit{TiVi} \cite{TiVi2021}, \textit{Agora} \cite{AgoraWhitePaper2021}, and \textit{Voatz} \cite{Voatz2021}. The remarkable difference between the nature of approaches regarding their real-world applications is a strong indicator of the potential of blockchain in this scenario. Real-world blockchain-based e-voting solutions follow the academic approach much closer than their centralised counterparts.
\par
The real-world solutions indicated are end-to-end applications, but other proposals are just protocols used to set up e-voting systems instead. These are not complete solutions, as the ones indicated thus far, but instead "recipes" that, if followed, can be used to establish a secure and transparent e-voting system. \cite{Lamtzidis2023} presents a concise summary of the most relevant Ethereum-based protocols in existence. Most of the logic employed in these protocols is already abstracted through smart contracts already deployed and publicly available in the Ethereum blockchain, such as the \textit{MACI (Minimal Anti-Collusion Infrastructure)} protocol \cite{MACI2024}, \textit{Semaphore} \cite{Semaphore2024}, \textit{Cicada}, and \textit{Plume}. It is important to notice that none of these protocols employs NFTs as an abstraction of votes as well. There are references to NFTs in the protocol description, but these are used to exemplify how the protocol handles ownership of digital objects or uses NFT ownership as a means to verify the identity of a voter, but never as the main data element carrying the voter's choices.
\par
To finalise this section, we attempted to review any e-voting proposals that were using NFTs explicitly in the voting process. Any usage of NFTs in any capacity in a remote voting system was considered relevant, yet our search was unable to find a single complete proposal that combined both. For this purpose, we consulted the main academic databases, namely \textit{Google Scholar}, \textit{Science Direct}, \textit{IEEE}, and \textit{ACM}. We started the search using a broader search term, namely, "e-voting" and "NFT," as well as expanded acronyms and other variations, without much success. The closest article to an NFT-based e-voting system we were able to find was \cite{Sagar2023}, where the authors use SoulBound NFTs, a special case of non-transferable NFTs that can potentially be used for identification purposes \cite{Weyl2022}, to circumvent the need for a trusted third party to implement \textit{eligibility}, i.e., if a voter is allowed to vote in a given election or not. The proposed system only interacts with these NFTs during the voter validation phase.
\par
\cite{Bistarelli2022} and \cite{Agbesi2019} do mention Non-Fungible Tokens, but they do it more as a product of their literature review rather than as an integral component in the solution. To conclude this search process, \cite{Ali2023}, \cite{Bao2022}, and \cite{Wang2021} produced extensive surveys around the potentials and usage of NFTs, as well as providing a list of future challenges where this technology can be determinant. \cite{Ali2023} does mention a potential application of NFTs in governance applications, but without specifying voting or even elections in any capacity. \cite{Wang2021} listed challenges limited to purely digital applications, namely gaming, virtual events, digital collectibles, and metaverse applications. \cite{Bao2022} provided a broader survey and identified a larger and more specified set of potential applications for NFTs, but none of them related to governance or e-voting.
\par
This analysis reveals a research gap regarding the application of NFTs in an e-voting context. Public blockchains hve been quite successful in deploying the NFT feature is a secure and transparent fashion. NFT are currently being transacted at a high rate in public blockchains and are able to do so securely and privately, which is the type of behaviour expectable from a trusted e-voting system.
\end{document}