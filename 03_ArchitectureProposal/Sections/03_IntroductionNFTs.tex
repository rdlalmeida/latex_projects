\documentclass[../3_VotingAndNFTs.tex]{subfiles}
\graphicspath{{\subfix{../../Images}}}

\begin{document}
\textit{Non-Fungible Tokens (NFTs)} followed cryptocurrencies as another example of digitally unique constructs, which also contribute to the concept of \textit{digital scarcity}, but with functional differences. As implied in the name, NFTs are not fungible, i.e., unlike Bitcoin or Ethereum cryptocurrency tokens, which are interchangeable and can be transacted as fractions of a unit, every NFT is digitally unique and can only be transferred whole. A user can have only a fraction of a Bitcoin in an account (0.56 BTC, for example), but the same is not valid for NFTs. Either a user owns it whole or not. Just as well, users can exchange any cryptocurrency tokens among themselves.
\par
The NFT concept is transversal to all blockchains, but since the concept was introduced through the Ethereum chain, the NFT standard is regulated by two \textit{Ethereum Improvement Proposals (EIP)}, namely EIP-721 and EIP-1155, to define a set of base requirements (variables and functions) that a smart contract needs to implement to conform to the standard. The \textit{Ethereum Request for Comments-721} (ERC-721) \cite{ERC721} and ERC-1155 \cite{ERC1155} standards regulate NFTs in Ethereum. It is possible for someone to define an NFT outside of these standards since they are not legally enforceable in any fashion. Yet, since the inception of this concept, the vast majority of published NFTs follow this standard since this gives them a level of default interoperability that is hard to achieve otherwise. If a given NFT implements the standards indicated, other users and developers have the guarantee, due to technological requirements that are "forcibly" implemented through the usage of these standards, that the variables and functions defined in the interface are implemented in the NFT contract, similar to what already happens with interfaces in object-oriented programming paradigms.
\par
The application potential of this technology regarding digital collectibles has in itself motivated the creation of NFT-centric blockchains, such as Flow \cite{Hentschel2019a}, which were developed towards overcoming aspects that make NFT mechanics too expensive, both in gas spent, resources allocated, and execution time, in more general-purpose blockchains such as Ethereum, for example, as well as online marketplaces dedicated solely to the commercialisation of NFTs (e.g., OpenSea \cite{OpenSea2024}) as long as the NFT smart contract implements the standards indicated.
\par
Unlike cryptocurrencies, NFTs can store metadata on-chain. Yet, because of the high cost associated with writing operations, in most cases, to optimise cost, most of the metadata stored in an NFT is a URL that can be automatically resolved to an off-chain resource, typically an image, a video, or any other type of digital file. This is the most common approach with artistic NFTs and even most digital collectibles \cite{Trautman2022}.

\subsection{NFTs vs. Cryptocurrencies}
In a blockchain context, NFTs represent objects, while cryptocurrencies are variables. Though most NFTs produced thus far represent \textbf{digital} objects, there are no strict requirements in that regard. NFTs can easily be used to represent physical objects, or more correctly, to establish ownership of those physical objects in a digital distributed ledger, but so far the emphasis has been on keeping everything in the digital realm. The amount of cryptocurrency owned by an account is determined by either a balance value stored in the governing contract or by determining the Unspent Transactional Output (UTxO) value associated with the account. Ethereum maintains a balance record while Bitcoin processes transactions UTxOs to determine account balances. Adding data to a blockchain using only cryptocurrencies is a challenge in itself, since these do not provide a direct mechanism to write arbitrary data into a block. Before the NFT standard, researchers looking to use a cryptocurrency-based blockchain for alternative applications had to be creative in that regard.
\par
Bitcoin was the only research alternative until the introduction of Ethereum in 2015. Bitcoin added new functionalities periodically, and its $0.9.0$ version introduced the \textit{OP\_RETURN} instruction to the execution set. This instruction always writes its input, unchanged, into the transactional data that gets written in the blockchain \cite{Bartoletti2017}, which provided researchers with a practical alternative to sending arbitrary data to a blockchain.
\par
NFTs and smart contracts provide much more efficient and flexible means to achieve the same result. As indicated, NFTs are defined as digital objects, i.e., a pre-defined data structure that can contain several internal parameters, or even other objects if needed. Minting NFTs writes their metadata into a block. Depending on the blockchain architecture, this data can be changed afterwards, but, typically, changes to the NFT metadata require a digitally signed transaction, which can only be produced by the owner of the account that owns that NFT.

\subsubsection{Token Burning}
Token ownership in a blockchain is abstracted by the ability of a user to access the containing account, namely, the capability of that user to generate a valid digital signature that can sign a transaction to transfer that token to some other location. Deriving an account address from a private encryption key is trivial, but the opposite is computationally infeasible. The blockchain establishes ownership mechanics through this asymmetrical relationship.
\par
Up to the point of this writing, there is no known private encryption key from which it is possible to derive a \textit{zero address}, i.e., an account address composed solely of zeroes (0x00000...), which implies that no one "controls" that address. Transferring a token to this unrecoverable address effectively "burns" it. Once a token goes into the \textit{zero address} account, no one can move it out of it due to the lack of a valid private key that can sign the required transaction. Therefore, "burnt" tokens are considered as if they were destroyed when in reality they are simply stored in an account that no one is able to control \cite{Antonopoulos2018}.

\end{document}