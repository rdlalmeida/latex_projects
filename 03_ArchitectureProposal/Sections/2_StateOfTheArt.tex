\documentclass[../main.tex]{subfiles}
\graphicspath{{\subfix{../Images}}}

\begin{document}
    The history of e-voting systems research details an evolution process that was triggered by the first proposals resulting from the "commercialisation" of cryptography, up to the latest decentralised proposals using blockchain as the technological basis.
    \par
    Cryptography has been used for some time but mainly within a military context, where secure communications are a critical element for battlefield success. Cryptography informally entered the public domain through the publication of \cite{Diffie1976}, which proposed the computation of logarithms over a finite field with a prime number of elements as the intractable problem required to produce cryptosystem. This was the first proposal for a non-military, non-proprietary cryptosystem, and soon after, other articles detailing alternative solutions and exploring other sources for intractable problems followed, such as \cite{Shamir1979}, \cite{Chaum1981}, \cite{ElGamal1984}, and \cite{Rivest1983}, which proposed the first commercial cryptosystems based on symmetrical and asymmetrical encryption keys. The applicability of these cryptographic tools in an e-voting context was fairly clear and the first e-voting proposals based on these concepts soon followed. These early systems are characterised by cryptographic tools derived from research on commercial cryptography, such as blind signatures \cite{Chaum1983}, Mix-Nets \cite{Chaum1988}, Homomorphism in threshold cryptosystems \cite{Shamir1979} and cryptographic proofs \cite{Goldwasser1986}.
    \par
    Research in e-voting systems progressed towards the establishment of a classification criteria that were then used to compare proposals from a security standpoint. Authors implemented criteria such as \textit{accuracy}, \textit{privacy}, \textit{eligibility}, \textit{verifiability}, \textit{convenience}, \textit{flexibility}, \textit{mobility} and \textit{robustness} using techniques obtained through research in commercial cryptography. A simple example that illustrates this process is the usage of asymmetrical encryption keys to encrypt voter data, thus protecting the \textit{privacy} of the voter. A proposal that uses such scheme can claim that it establishes voter \textit{privacy}. Yet, a formal definition of such criteria has notorious absent from related literature. \cite{Neumann1993} was among the first to attempt such characterisation, with subsequent publications, such as \cite{Fujioka1992}, \cite{Baraani1995}, \cite{Juang1997}, \cite{Ku1999}, \cite{Lee2000}, \cite{Joaquim2003}, \cite{Baiardi2005}, and \cite{Chaum2007}, continuing this trend. These articles followed the rationale that the more secure a system is, i.e., the more security criteria it implements, the more it can assure a user that he/she can trust his/her choice to it. Over time, these cryptographic tools became a fixture in all e-voting proposals in this initial 30-year window of research in centralised e-voting systems. This is a limiting paradigm since it constrains the whole system by establishing a single point of attack or failure, while also reducing system scalability, due in great part to the demand of a large amount of resources, such as primary and secondary storage, computational power, network bandwidth, etc., to implement these criteria.
    \par
    Scalability is an important characteristic that can hinder a wide adoption of the proposed system. Contemporary elections can go from simple exercises, where the system is expected to process up to a thousand votes at one point, to national-wide events that require the processing of millions of votes instead. The relationship between the scalability of a system and the amount of available resources is quite evident in the analysed literature. Proposals from this era confirm that the ones that satisfy the most security criteria also establish a computationally complex and demanding system that is often limited to small-scale elections. As an example, in \cite{Cranor2002}, \cite{Nurmi1991}, \cite{Iversen1992}, and \cite{Niemi1999}, this trade-off was shifted towards security and transparency at the expense of scalability; hence, these systems are all limited to small-scale elections, as the authors declare.
    \par
    On the other side of this spectrum, proposals such as \cite{Benaloh1994}, \cite{Benaloh1986a}, \cite{Park1994}, \cite{Juang2002}, \cite{Okamoto1996}, \cite{Okamoto1998}, \cite{Magkos2001} or \cite{Moran2006} present scalable systems that are significantly simpler than the ones considered in the last paragraph, but their adoption in large-scale elections results in a substantial sacrifice in security and transparency since these implement the lowest number of security criteria considered.
    \par
    Several e-voting systems were evaluated in a real-world scenario. For this case, we are only interested in systems that address the criterion of \textit{mobility}, i.e., a voting system that does not restrict voters geographically, in part because these proposals did not follow any of the academic ones that preceded them. Therefore, there was little interest in electronic proposals that limited their users to traditional polling places, since they infuse a degree of privacy and security that derives solely from the surrounding election apparatus. The few notable exercises in recent history were run in Canada in 2013 \cite{Goodman2014}, Estonia in 2005 \cite{Heiberg2005}, Switzerland in 2005 \cite{Binder2019}, Norway in 2011 \cite{Barrat2012}, France in 2012 \cite{Pinault2012}, and Australia in 2015 \cite{Halderman2015}.
    \par
    The main discovery from this analysis is an overlap between academic proposals and real-world systems, or, more precisely, the lack thereof. Private companies developed most of the real-world remote solutions indicated by being awarded government development contracts. As a consequence, specific technical details are kept private as patented intellectual property, with any technical information about them available only from review reports, which makes any further analysis a difficult endeavour. The complexity and resource hungriness of centralised solutions are characterised by the requirement of government-scale investments to achieve minimal implementation. In large-scale elections, the scalability effort quickly grows out of the capacity of small organisations and individuals, leaving governments as the only organisations with enough resources to implement them. This generation of e-voting systems proved that the concept is sound and that it is possible to use electronic means to establish elections with greater accuracy and security for their participants.
    \par
    A shift towards a decentralised approach to this issue initiated with proposals that used blockchain merely for transport and record, in part as a consequence of the "imposition" of using Bitcoin's blockchain, since it was the only mature public blockchain available at the time. Some proposals \cite{Zhao2016}, \cite{Cruz2016}, \cite{Bistarelli2017}, \cite{Lee2017}, \cite{Shaheen2017}, \cite{Wu2017}, \cite{Dimitriou2020} or \cite{Bartolucci2018} used a script function available in Bitcoin transactions to add voting information to the blockchain data, namely the \textit{OP\_RETURN} function, which receives an 83-byte wide string as input, and adds it to the transaction metadata as the function's output (depends on the version of the Bitcoin protocol. The last one extended the size of the allowed output to 83 bytes. Earlier versions are more limited in size). Hence, it was used to write non-transactional data directly to the blockchain. This mechanism needs to go around the limited functionalities offered by early blockchain solutions whose scope of operations were limited to cryptocurrency transactions only. Furthermore, this method is infeasible for large-scale use because a Bitcoin transaction necessarily involves exchanges worth a considerable amount of money, as well as being notoriously hard-to-scale blockchain due to its high block rate. Bitcoin adds a new block every 10 minutes, which limits severely the rate of operations that this blockchain can withstand.
    \par
    The popularisation of public blockchains attracted interest from other platforms, which triggered the development of software frameworks used to create and deploy custom blockchains with proprietary access control and offering more flexibility to applications. As such, researchers deployed custom-made solutions in customisable frameworks such as Hyperledger Fabric, Quorum, and Multichain. Some proposals, \cite{Kirby2016}, \cite{BenAyed2017}, \cite{Chaieb2018}, \cite{Burhanuddin2018}, \cite{Zhang2018}, \cite{Khan2018}, \cite{Murtaza2019}, \cite{Faour2019}, \cite{Killer2020}, \cite{Han2020}, \cite{Vivek2020}, \cite{Mani2022}, \cite{Zhou2020}, \cite{Alvi2022}, \cite{Hassan2022}, \cite{Vidwans2022} or \cite{Matile2019} adopted private blockchains tailored to the voting application. As a drawback, the increased flexibility is paid for by a lack of network support. It is difficult to establish a privately accessible network with enough active nodes that can establish a satisfactory level of redundancy.
    \par
    A significant breakthrough arrived with the introduction of the smart contract through the Ethereum blockchain, specifically through the implementation of a \textit{Touring-complete} processing platform, named \textit{Ethereum Virtual Machine (EVM)}, that can execute code scripts in a decentralised fashion by splitting and distributing the instructions through the active nodes in the network. This opened up new research avenues in the e-voting environment. A few years after Ethereum's debut, the first proposals used smart contracts to establish the logic of the system in a decentralised approach, such as \cite{McCorry2017}, \cite{Koc2018}, \cite{Dagher2018}, \cite{Fusco2018}, \cite{Hjalmarsson2018}, and \cite{Mols2020}.
    \par
    Some researchers did base their solution on the Ethereum network, but they did not resort to smart contracts; instead, they explored the increased flexibility and additional application support of this network, employing similar methods as the ones in earlier proposals. Examples of this strategy can be found in \cite{Hardwick2018}, \cite{Wang2018}, \cite{Hsiao2018}, \cite{Lai2018}, \cite{Shukla2018}, and \cite{Vo-Cao-Thuy2019}.
    \par
    The same time period saw several blockchain-based proposals for electronic voting systems in a real-world scenario. Small organisations are putting forward these proposals. But unlike the centralised approach, these solutions have significant overlap with the academic proposals considered. Among these real-world examples, we cite \textit{Follow My Vote} \cite{Ernest2021}, \textit{TiVi} \cite{TiVi2021}, \textit{Agora} \cite{AgoraWhitePaper2021}, and \textit{Voatz} \cite{Voatz2021}. The remarkable difference between the nature of approaches regarding their real-world applications is a strong indicator of the potential of blockchain in this scenario. Real-world blockchain-based e-voting solutions follow the academic approach much closer than their centralised counterparts.
    \par
    The real-world solutions indicated are end-to-end applications, i.e., they are ready to be used in an election, as long as their limitations are properly addressed (mostly related to scalability). But there are other proposals published in the public domain as protocols that can be used to set up an e-voting systems instead. These are not complete solutions, as the ones indicated thus far, but instead "recipes" that, if followed, can be used to establish a secure and transparent e-voting system. \cite{Lamtzidis2023} presents a concise summary of the most relevant Ethereum based protocols in existence. Most of the logic employed in these protocols is already abstracted through smart contracts already deployed and publicly available in the Ethereum blockchain, such as the \textit{MACI (Minimal Anti-Collusion Infrastructure)} protocol \cite{MACI2024}, \textit{Semaphore} \cite{Semaphore2024}, \textit{Cicada}, and \textit{Plume}. It is important to notice that none of these protocols employs NFTs as an abstraction of votes as well. There are references to NFTs in the protocol description, but these are used to exemplify how the protocols handles ownership of digital objects or using NFT ownership as means to verify the identity of a voter, but never as the main data element carrying the voter's choices.
    \par
    To finalise this section, we attempted to review any e-voting proposals that were using NFTs explicitly in the voting process. Any usage of NFTs in any capacity in a remote voting system was considered relevant, yet our search was unable to find a single complete proposal that combined both. For this purpose, we consulted the main academic databases, namely \textit{Google Scholar}, \textit{Science Direct}, \textit{IEEE}, and \textit{ACM}, using a broader search term at first, namely, "e-voting" and "NFT," as well as with expanded acronyms and other variations, without much success. The closest article to a NFT-based e-voting system we were able to find was \cite{Sagar2023}, where the authors use SoulBound NFTs, a special case of non-transferable NFTs that can potentially be used for identification purposes \cite{Weyl2022}, to circumvent the need for a trusted third party to implement \textit{eligibility}, i.e., if a voter is allowed to vote in a given election or not. The proposed system only interacts with these NFTs during the voter validation phase. The article does not provide enough technical details to determine exactly how blockchain is used in the remainder of the voting process, but it is clear with regard to where and when they use these NFTs.
    \par
    Two other proposals, namely \cite{Bistarelli2022} and \cite{Agbesi2019}, do mention Non-Fungible Tokens, but more so as a product of their literature and technology review and not as an integral component in their solution. To conclude this search process, \cite{Ali2023}, \cite{Bao2022}, and \cite{Wang2021} produced extensive surveys around the potentials and usage of NFT, as well as providing a list of future challenges where this technology can be determinant. \cite{Ali2023} does mention a potential application of NFTs in governance applications, but without specifying voting or even elections in any capacity. \cite{Wang2021} listed challenges limited to purely digital applications, namely gaming, virtual events, digital collectibles, and metaverse applications. \cite{Bao2022} provided a broader survey and identified a larger and more specified set of potential applications for NFTs, but none of them related to governance or e-voting.
    \par
    As far as we were able to determine, no proposals were submitted thus far using NFTs as an integral element of an e-voting system.
\end{document}