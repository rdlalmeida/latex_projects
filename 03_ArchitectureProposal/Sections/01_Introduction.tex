\documentclass[../main.tex]{subfiles}
\graphicspath{{\subfix{../Images}}}

\begin{document}
Modern democracies are critical to technological advances, as they provide citizens with levels of comfort and security that allow them to divert their time and energy away from basic survival towards research and innovation. This is a transversal effect across society, with all sorts of mundane activities getting progressively easier and even automated, with each technological advance.
\par
Yet, the exercise that keeps these same modern democracies running smoothly still resists innovation. Most elements that characterise a modern society—banking, farming, food processing, education, healthcare, etc.—have improved thanks to technology. Except voting. Apart from a few minor upgrades on the voting act itself, the fundamental mechanics have remained mostly unchanged.
\par
From dropping clay shards in a pot to selecting an option on a touchscreen, there has not been a change significant enough to bring this process to the same levels of practicality as other activities. Votes submitted in an electronic medium can be counted faster, but other than that, current electronic voting still limits voters geographically and does not contribute to making elections easier or cheaper to organise.
\par
This is not an indication of a lack of interest in the research community. In fact, the modernisation of voting systems has been an active area of research in general, and the digitalisation of voting systems has been researched actively since the advent of commercial cryptography. A landmark article by Diffie and Hellman in 1976 \cite{Diffie1976} established the basis for commercial cryptography, a research area that was confined to governmental (i.e., military) applications until then. This publication spearheaded the development of several cryptographic tools, particularly schemes for symmetrical and asymmetrical encryption systems. Among the earlier proposals for asymmetrical cryptosystems were the \textit{Rivest-Shamir-Adleman (RSA)} \cite{Rivest1983} cryptosystem, which relies on the factorisation of large prime numbers, and the \textit{ElGamal} \cite{ElGamal1984} cryptosystem, which uses the intractability of computing discrete algorithms.
\par
In the specific context of e-voting, these advancements were followed by approximately three decades of e-voting research based on a centralised, server-client paradigm that produced a significant number of academic proposals that used cryptographic tools such as encryption, digital signatures, one-way hash functions, etc., to create secure and transparent e-voting systems.
\par
This was the state of e-voting research until 2009. At the end of that year, blockchain broke into the mainstream through Bitcoin, the first cryptocurrency, introduced through a landmark paper \cite{Nakamoto2008}. After Bitcoin's solidification as the first truly digital currency, others followed, improving not only the currency feature but also adding support to different applications. Six years later, the Ethereum blockchain \cite{Dannen2016} was launched, and with it the concept of a blockchain virtual machine, a functional abstraction of the cumulative resources available in the network nodes, which could be used to run code synchronously—referred to as \textit{smart contracts}—in a subset of the network nodes. Ethereum's version of this virtual machine is named \textit{Ethereum Virtual Machine (EVM)} \cite{Antonopoulos2018}. Though Bitcoin already offered some capacity to execute scripted code in a distributed manner, this was not its main feature and thus was quite limited in that regard. Ethereum was the first blockchain to offer explicit support for smart contracts, but other public blockchains created after followed it in this regard. Smart contract support is a base feature for most public blockchains currently, which has made this concept almost ubiquitous in the blockchain ecosystem. Other examples of blockchains with smart contract support created after Ethereum include \textit{Cardano}, \textit{Solana}, \textit{Tron}, \textit{EOS}, \textit{Polkadot}, and \textit{Flow} \cite{cointree2022}.
\par
The applicability of blockchain to the e-voting context was clear from early on, and e-voting research migrated to this new approach to take advantage of these features. The first academic articles detailing e-voting systems using a blockchain appeared in 2016, and since then, the interest in this new research approach has increased steadily.
\par
This article presents a fully mobile, blockchain-based electronic voting system that explores the concept of Non-Fungible Tokens (NFTs), one of the newest features put forward by the Smart Contract-enabled Ethereum network. We discuss this concept in greater detail in Section \ref{introduction_nfts}.
\par
This paper continues with a state of the art in Section \ref{stateoftheart}, where we discuss the evolution of e-voting systems research. Section \ref{general_approach} details a generalist approach to the problem at hand and is followed by a description of the two specific approaches considered in Sections \ref{contract-based-approach} and \ref{account-based-approach}. A security analysis of the proposed solution follows with Section \ref{features_and_security}. Section \ref{conclusion} concludes the paper with a final conclusion and a discussion over open problems.
\end{document}