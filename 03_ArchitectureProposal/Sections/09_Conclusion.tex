\documentclass[../main.tex]{subfiles}
\graphicspath{{\subfix{../Images}}}

\begin{document}
Electronic voting systems are still an active area of research and a playground from which some important cryptographic concepts derive. This has transitioned e-voting research to a new age regarding the fundamental approach considered, thanks to its recent integration with blockchain technology.
\par
This proposal presents a decentralised, remote electronic voting system that takes advantage of the properties of non-fungible tokens, as well as the Flow blockchain and its Resource-Oriented Paradigm. The organisation and features offered by this new approach support the design of a private, secure, transparent, verifiable, and mobile voting system whose proof of concept is to be derived from the strategy presented in this text.
\par
This also proves the suitability of the NFT concept as a transport mechanism within public networks.
\par
This paper defines a concrete roadmap to produce a working prototype towards a secure and remote e-voting system. Another important aspect that determines the usability of this proposal is scalability; specifically, what are the limits of the supporting blockchain regarding the scope of operation of this system?
\end{document}