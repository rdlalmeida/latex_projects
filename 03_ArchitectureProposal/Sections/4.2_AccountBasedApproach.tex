\documentclass[./4_GeneralApproach.tex]{subfiles}
\graphicspath{{\subfix{../../Images}}}

\begin{document}
    Fig. \ref{fig:account_based_architecture} displays the proposal system under an account-based blockchain environment. 
                
    \begin{figure}[htp]
        \centering
        \includegraphics[width=0.7\textwidth]{../Images/03_account_based_solution.png}
        \caption{Account-based version of the e-voting system proposed.}
        \label{fig:account_based_architecture}
    \end{figure}

    The main difference with this approach is that, in this case, the Vote NFT is always stored under the account of the element that holds the token at a given point in the process, i.e., the Vote NFT is actually moved through distinct storage locations throughout the process, which are abstracted and referenced as accounts, as opposed to the contract-based approach, where the only changes are to the internal mappings of the NFT-generating contract, typically indicating which address controls which NFT. From a voter's perspective, he or she should not detect any changes in the voting experience.

    \subsubsection{Account-based Blockchain to use}
    For this case, we considered the Flow blockchain for two main reasons:
        \begin{enumerate}
            \item{Flow \cite{Hentschel2019a} was developed by the creators of CryptoKitties \cite{Gharegozlou2019}, one of the first examples of an NFT-generating smart contract in the Ethereum network. Besides establishing one of the first real-world examples of digital collectibles, the CryptoKitties smart contract established a series of game-like characteristics designed to attract users to experiment with the concept.
            \par
            The success of this initiative exposed the potential behind NFT technology as well as how limited and difficult it really was to scale the Ethereum blockchain. The network experienced difficulties in responding to an unusually high volume of transactions due to the rapid popularity of CryptoKitties.
            \par
            Flow was developed as a scalable, resource-based blockchain. Like many public blockchains developed in recent years, it also developed Cadence \cite{Cadence2023}, an interpretative programming language used to create smart contracts, as well as expanding the functionalities of this blockchain through automated Cadence-written scripts and transactions.}

            \item{Flow is a good example of an account-based blockchain. Given the resource-oriented nature of this blockchain as well as its claims of scalability, there was some emphasis on data storage when planning this blockchain, which is very useful to the system we intend to implement.}
        \end{enumerate}

    \subparagraph{Account model in Flow}
    \label{flow_accounts}
    Flow blockchain implements accounts as a record of a specific state of the blockchain defined by:
    \begin{itemize}
        \item{An unique address consisting of an 8 byte or 16 hex-encoded character string that can optionally be preceded by \textbf{0x} to indicate the format used (optional).}
        \item{A set of public keys that are required for any interaction with that account, as well as to verify the validity of digital signatures. Unlike Ethereum and other blockchains, Flow supports multiple public keys in a single account, which enables multi-user-controlled accounts. Keys in the public set can be created with different weights to allow multi-signature schemes that also allow for non-homogeneous distribution of account control \cite{flow2024}.}
        \item{Smart contract code, which is saved in a specific dedicated area of the storage space. A single account can store multiple contracts.}
        \item{The main storage where all digital objects (resources) are stored.}
    \end{itemize}

    Fig. \ref{fig:flow_account_model} illustrates this organization.

    \begin{figure}[htp]
        \centering
        \includegraphics[width=0.7\textwidth]{../Images/Flow_account_model.png}
        \caption{Overview of the account model used in the Flow blockchain. Source: \cite{flow2024}}
        \label{fig:flow_account_model}
    \end{figure}

    An important fact to retain from Fig. \ref{fig:flow_account_model} is that contracts in this model are stored in a distinct area from general-purpose storage and, as such, have different access rules. Like with any other smart contract-capable blockchain, smart contracts in Flow can be read by anyone, even though they are stored in an account-specific storage area.

    \paragraph{How storage works on Flow}
    Flow stores data as digital objects, which are referenced as resources. This aspect is central to the functionality of this blockchain, so much so that its developers define the programming paradigm imposed by Cadence as a "resource-oriented paradigm" \cite{flow2024}.
    \par
    Storage in the Flow blockchain is account-based, which means that all digital objects permanently stored in this blockchain are referenced by the controlling account. The internal mechanics of the Flow blockchain ensure that only the owner of an account can alter its storage, namely, read, write, and manipulate the access control (permissions) associated with data objects in storage. Other than contracts, Cadence is used to write executable scripts and transactions. Functionally, these are similar, but with one exception: modifications to the state of the blockchain (add or remove resources, token transfers, etc.) can only be performed through transactions, and these, as is the rule in all blockchains, need to be digitally signed by the account owner. Scripts are "lighter" versions of transactions that do not require a digital signature to be executed and are restricted to reading data only. This guarantees that only the owner of the account is able to manipulate the data in it.
    \par
    Similar to all other blockchains, storage space in Flow is not free, though it is arguably cheaper than in other popular blockchains such as Ethereum. The amount of storage allocated to an account is proportional to the balance of the FLOW token (Flow's native cryptocurrency, from here referred to in all capitals to distinguish it from the main blockchain) of the account in question, establishing a rate of 100MB of storage space per FLOW token, i.e., a user with a balance of 1 FLOW in his/her account can store up to 100MB of data in storage \cite{flow2024}. If an operation (transaction or smart contract function) stores data above the allowed capacity of an account based on the current balance, the transaction fails and all state changes are reverted, just like with a regular smart contract exception.
    \par
    Flow also establishes a base value of 100 kB of storage space that remains available in perpetuity. This establishes two consequences:
    \begin{enumerate}
        \item {A user needs to pay to create an account. The "permanent" storage costs 0.001 FLOW, according to the rate mentioned above, and has to be paid during account creation, though these tokens actually end up in the main account balance.}
        \item {Flow accounts require a minimum balance of 0.001 FLOW at all times to pay for this permanent storage, therefore Flow accounts cannot be completely emptied \cite{flow2024}.}
    \end{enumerate}

    \subparagraph{Storage domains}
    \label{storage_domains}
    Each account in Flow has three domains that define its storage space: \textit{/storage}, \textit{/public}, and \textit{/private} \cite{flow2024}. These are indicated as UNIX-style paths to reinforce the notion that they are mainly used to indicate where a digital object is stored in the storage space allocated to a specific account. This account storage is only used to save digital objects as well as references (akin to memory pointers. Please refer to Section \ref{capabilities_references} for additional details), used to delegate read access to stored objects without risking their modification or deletion from other users and/or smart contracts.
    \par
    All resources stored in an account's storage go into the \textit{/storage} domain by default. Only the account owner(s) have access to this domain, for both write and read privileges. This means that any resource transferred into an account, either through a transaction or a functionally equivalent smart contract function, becomes private by default as soon as it is stored. To allow for external access, the controlling user must move that resource to either the \textit{/public} or \textit{/private} domains. Data in the \textit{/public} domain is available to everyone. The \textit{/private} domain, despite the name, actually sits somewhere between the \textit{/storage} and \textit{/public} domains regarding access. Data stored in this domain is only available to the owner and to users previously authorised by the owner.            

    \subparagraph{Capabilities}
    \label{capabilities_references}
    \textit{Capabilities} are tools used in the Flow blockchain to delegate access to resources (data objects). An account controller creates \textit{capabilities}, which are akin to a memory pointer in that instead of referencing a memory position, they reference a specific resource stored in the account where the \textit{capability} was created. \textit{Capability} creation changes the state of the blockchain; therefore, it needs to be done in a transaction digitally signed by the account owner to be executed. These requisites restrict the creation of \textit{capabilities} to the owner of the resource in question.
    \par
    Flow distributes \textit{capabilities} among account holders using a messaging system. Once created, the account owner publishes the \textit{capability} to an account address to delegate control of the target resource. The \textit{capability} is delivered to the inbox of the address indicated, to be retrieved at a later point by the target account owner.
    \par
    Interfaces, a recurring concept in both general-purpose programming languages, Solidity and Cadence, are used to restrict the visibility of a resource. Visibility in this context means the internal parameters, events, and functions that remain available once access to the resource is delegated through a \textit{capability}. Cadence uses interfaces to define mandatory standards for classes, similarly as in Solidity, but with an additional layer of access control for future instances of that class in storage.
\end{document}