\documentclass[../NFTComp_IEEE.tex]{subfiles}
\graphicspath{{\subfix{../Images}}}

\begin{document}
\section{Results and Comparison}
\label{sec:architecture_comparison}

\subsection{Blockchain Comparison}
The Flow blockchain is significantly different from general purpose blockchains available publicly. To reinforce this conclusion, we took Ethereum, the most popular blockchain used in the academic context, as the reference point to produce Table \ref{tab:ethereum_flow_comparison_table} where the two solutions were compared side by side regarding the most relevant characteristics towards NFT implementations.

\begin{table}[ht]
    \scriptsize
    \caption{Ethereum-Flow Technology comparison}
    \centering
    \begin{adjustwidth}{0cm}{}
        \begin{tabular}{@{} m{2.5cm} ll@{}}
            \toprule
            \multicolumn{1}{c}{\multirow{2}{*}{\textbf{Parameter}}}      & \multicolumn{2}{c}{\textbf{Non-Fungible Token Architecture}}                                                       \\ \cmidrule(l){2-3}
            \multicolumn{1}{c}{}                                         & \multicolumn{1}{l}{\textbf{Ethereum}}                               & \textbf{Flow}                                \\ \midrule
            \textbf{Year}                                                & \multicolumn{1}{l}{2013}                                            & 2020                                         \\ \midrule
            \textbf{Cryptocurrency}                                      & \multicolumn{1}{l}{ETH}                                             & FLOW                                         \\ \midrule
            \parbox[m]{2.5cm}{\textbf{Programming Language}}             & \multicolumn{1}{l}{Solidity}                                        & Cadence                                      \\ \midrule
            \parbox[m]{2.5cm}{\textbf{Consensus                                                                                                                                               \\Algorithm}}                                 & \multicolumn{1}{l}{\parbox[m]{2.2cm}{2013-2022: PoW,                                            \\2022-: PoS}} & Proof-of-Stake (PoS)                       \\ \midrule
            \multirow{4}{*}{\parbox[m]{2.5cm}{\textbf{Network Nodes                                                                                                                           \\Roles Types}}} & \multicolumn{1}{l}{\multirow{2}{*}{1 - Execution Client}}          & 1 - Collector Node                   \\ \cmidrule(l){3-3}
                                                                         & \multicolumn{1}{l}{}                                                & 2 - Consensus Node                           \\ \cmidrule(l){2-3}
                                                                         & \multicolumn{1}{l}{\multirow{2}{*}{2 - Consensus Client}}           & 3 - Execution Node                           \\ \cmidrule(l){3-3}
                                                                         & \multicolumn{1}{l}{}                                                & 4 - Verification Node                        \\ \midrule
            \multirow{2}{*}{\parbox[m]{2.5cm}{\textbf{Token Standards}}} & \multicolumn{1}{l}{\parbox[m]{2.4cm}{ERC-20 (fungible token)}}      & FungibleToken                                \\ \cmidrule(l){2-3}
                                                                         & \multicolumn{1}{l}{\parbox[m]{2.4cm}{ERC-721 (non-fungible token)}} & NonFungibleToken                             \\ \midrule
            \parbox[m]{2.5cm}{\textbf{Data Storage                                                                                                                                            \\Architecture}}   & \multicolumn{1}{l}{Contract-based}                                 & Account(User)-based                  \\ \midrule
            \textbf{Block Rate (Average)}                                & \multicolumn{1}{l}{\parbox[m]{2.0cm}{12 - 15 seconds per block}}    & \parbox[m]{2.0cm}{0.5 - 1 seconds per block} \\ \midrule
            \parbox[m]{2.5cm}{\textbf{Daily transaction volume (2024)}}  & \multicolumn{1}{l}{\parbox[m]{2.0cm}{1 - 1.25 million                                                              \\transactions\textbackslash day}}      & \parbox[m]{2.0cm}{0.5 - 1 million\\transactions\textbackslash day} \\ \midrule
            \parbox[m]{2.5cm}{\textbf{Cost per transaction (average)}}   & \multicolumn{1}{l}{5.5 gwei ($\sim$0.39 \$)}                        & $\sim$0.00000845 \$                          \\ \bottomrule
        \end{tabular}
    \end{adjustwidth}
    \label{tab:ethereum_flow_comparison_table}
\end{table}

The two technologies are similar in their general blockchain approach: both implement a cryptocurrency token and use it as a regulatory mechanism to optimise the functioning of their respective virtual machines. Both implement the concept of gas, a fraction of the cryptocurrency implemented, that needs to be paid beforehand to execute transactions that change the state of the blockchain. Ethereum coined the term \textit{gas} to refer to cryptocurrency used to pay virtual machine computational costs. Flow implements the same exact logic, though they do not use the \textit{gas} term as much. Both blockchains use PoS as the consensus protocol, tough this is a recent upgrade for Ethereum. This chain spent the first 9 years using PoW, like Bitcoin, but it went through the Paris fork in September 2022, which implemented the \textit{Ethereum Improvement Proposal 3675 (EIP-3675)} \cite{EIP3675} that switched the consensus protocol to PoS from there on.
\par
The similarities stop here. From here, Flow and Ethereum provide very different programming languages for smart contract development. Both languages produce code that can be executed in the respective distributed virtual machines but using different syntaxes, albeit producing elements with similar behaviours and functionalities. Flow claims that its higher throughput, low operational costs and higher scalability are due to their four node pipelining architecture. Essentially Flow splits the two-node architecture from Ethereum into half, doubles the roles in the blockchain but this translates into a more efficient, but also more complex, blockchain. Both blockchains provide official standards to regulate token based development and mechanics, but the obvious incompatibility between the technologies produced a pair of token standards for each chain, among others.
\par
Flow is the cheapest and fastest of both. Flow's block rate is between 12 and 30 times higher than Ethereums and Ethereum transactions are, exactly 46154 times more expensive than Flow's. Yet, Ethereum's popularity and maturity still trump over Flow's operational advantages. At the time of this writing, Ethereum daily volume of transactions can be as 2.5 times higher than in Flow while being objectively worse to operate. This is not an endorsement of Ethereum, or Flow for that matter, but a factual statement: popularity and application ecosystem maturity influence technology adoption more that efficiency and cost, at least in a short term basis. It is important to notice that Ethereum has been available for twice as much time as Flow, and Flow made its debut in a time where many other blockchains were being created as well. Flow has yet to distinguish itself from other, more established NFT supporting blockchains. But from the performance point of view, the superiority of Flow is clear, especially when compared with the "older" technology of Ethereum.

\subsection{Contract Implementation Details}
We developed two smart contracts implementing the NFT standard in both Ethereum and Flow, i.e., implementing the \textit{ERC-721} and \textit{NonFungibleToken} standards respectively. Both contracts were used after to execute the following sequence of macro actions:
\begin{enumerate}
    \item {Deploy the NFT contract in a network}
    \item {Mint a NFT into a user account}
    \item {Transfer the NFT from one user account to another}
    \item {Burn the NFT}
\end{enumerate}

This sequence of actions were relatively straightforward to execute in the Solidity case. In Flow, there was some overhead to deal with first. Other than deploying a large contract, account creation is the most expensive atomic operation \cite{flow2024d}, both in gas and storage costs. Ethereum does not even has such function because of how fundamentally different Flow and Ethereum accounts are, but give the effort required to create one, we included this configuration step in the Flow analysis. Also, as it was indicated in Sec. \ref{sec:resource_collections}, Flow uses collection resources to simplify NFT storage. Though it is possible to save a NFT directly into a storage path, we opted to include this step as well to illustrate the impact that these overhead actions common in Flow have in the complete process.

\subsection{Cost Comparison}
\label{sec:cost_comparison}
We begun the comparison exercise by analysing the operational costs associated to each operation indicated, namely, how much gas was consumed per each step considered. Obtaining the gas spent for a specific set of steps is trivial. Ethereum development frameworks often include gas calculation tools that essentially keep track of the balance of the accounts involved in the process and computes the differences. We used Hardhat as the Ethereum development framework and the built-in \textit{gas reporter} tool to determine the gas expenditure indicated in Table \ref{tab:solidity_gas_reporter}:

\begin{table}[ht]
    \centering
    \caption{Gas consumption report from Hardhat's gas reporter tool.}
    \vspace{0.1cm}
    \label{tab:solidity_gas_reporter}
    \resizebox{\columnwidth}{!}{%
        \begin{tabular}{|ll|ll|l|ll|}
            \hline
            \multicolumn{2}{|l|}{Solc version: 0.8.24} & \multicolumn{2}{l|}{\multirow{2}{*}{\parbox[m]{1.8cm}{Optimizer enabled: false}}} & \multirow{2}{*}{Runs: 200} & \multicolumn{2}{l|}{\multirow{2}{*}{\parbox[m]{2.5cm}{Block limit: 6718946 gas}}}                                                                          \\ \cline{1-2}
            \multicolumn{2}{|l|}{Methods}              & \multicolumn{2}{l|}{}                                                             &                            & \multicolumn{2}{l|}{}                                                                                                                                      \\ \hline
            \multicolumn{1}{|l|}{Contract}             & Method                                                                            & \multicolumn{1}{l|}{Min}   & Max                                                                               & Avg                        & \multicolumn{1}{l|}{\# calls} & eur (avg) \\ \hline
            \multicolumn{1}{|l|}{ExampleNFT}           & burn(uint256)                                                                     & \multicolumn{1}{l|}{-}     & -                                                                                 & 29592                      & \multicolumn{1}{l|}{1}        & 0.8091    \\ \hline
            \multicolumn{1}{|l|}{ExampleNFT}           & safeMint(address, uint256, string)                                                & \multicolumn{1}{l|}{-}     & -                                                                                 & 121859                     & \multicolumn{1}{l|}{2}        & 3.3312    \\ \hline
            \multicolumn{1}{|l|}{ExampleNFT}           & \parbox[m]{3.8cm}{safeTransferFrom(address, address, uint256)}                    & \multicolumn{1}{l|}{-}     & -                                                                                 & 58401                      & \multicolumn{1}{l|}{2}        & 1.5973    \\ \hline
            \multicolumn{2}{|l|}{Deployments}          & \multicolumn{1}{l|}{-}                                                            & -                          & 2420549                                                                           & \multicolumn{1}{l|}{36 \%} & 66.205                                    \\ \hline
            \multicolumn{2}{|l|}{ExampleNFT}           & \multicolumn{2}{l|}{Totals}                                                       & 2630401                    & \multicolumn{1}{l|}{}                                                             & 71.943                                                                 \\ \hline
        \end{tabular}%
    }
\end{table}

We configured Hardhat's \textit{gas reporter} to connect to a cryptocurrency pricing oracle, an external application to the blockchain that can be queried for off-chain data, to obtain the gas costs in EUR in real time. ETH price currently is very volatile, but the values obtained are enough to illustrate the difference, especially compared to a much cheaper option of Flow. This notion was already presented in the last row in Table \ref{tab:ethereum_flow_comparison_table}, where the average price difference is quite apparent. Flow's development framework does not offer a comparable tool to Hardhat's \textit{gas reporter}. As such we calculated these fees directly by subtracting account balances between steps. Additionally, Flow imposes a base dependency to all standardised contracts to a special contract named \textit{FlowFees}. This contract is implemented in every Flow standard, including the \textit{NonFungibleToken} standard used, to automates a series of fee based computations and to emit a \textit{FeesDeducted} event every time any fees are payed while executing instructions from the contract that implements the standard. The event is emitted with the amount paid as argument and we also captured these events to validate the balance differences calculated. We used these strategies to determine the gas costs for the previous exercise with the Cadence contract. The calculations are indicated in Table \ref{tab:FLOW_balance}:

\begin{table}[ht]
    \centering
    \caption{FLOW token balance of each account in the exercise}
    \vspace{0.1cm}
    \label{tab:FLOW_balance}
    \resizebox{\columnwidth}{!}{%
        \begin{tabular}{lllllll}
            %--------------------------------------------------------------------------- GAS USED --------------------------------------------------------------------------------------------------------

            \hline
            \multicolumn{7}{|c|}{\textbf{FLOW token balance of accounts}}                                                                                                                                                                                                                                                                  \\ \hline
            \multicolumn{1}{|c|}{\multirow{2}{*}{\textbf{Tx}}} & \multicolumn{2}{c|}{\textbf{Emulator-account}}                                     & \multicolumn{2}{c|}{\textbf{account01}} & \multicolumn{2}{c|}{\textbf{account02}}                                                                                                    \\ \cline{2-7}
            \multicolumn{1}{|c|}{}                             & \multicolumn{1}{l|}{Balance}                                                       & \multicolumn{1}{l|}{Difference}         & \multicolumn{1}{l|}{Balance}            & \multicolumn{1}{l|}{Difference} & \multicolumn{1}{l|}{Balance} & \multicolumn{1}{l|}{Difference} \\ \hline \hline
            \multicolumn{1}{|l|}{\textbf{00}}                  & \multicolumn{1}{l|}{9.99600}                                                       & \multicolumn{1}{l|}{}                   & \multicolumn{1}{l|}{0}                  & \multicolumn{1}{l|}{}           & \multicolumn{1}{l|}{0}       & \multicolumn{1}{l|}{}           \\ \hline
            \multicolumn{1}{|l|}{\textbf{01}}                  & \multicolumn{1}{l|}{9.99396}                                                       & \multicolumn{1}{l|}{-0.00204}           & \multicolumn{1}{l|}{0.001}              & \multicolumn{1}{l|}{0.00100}    & \multicolumn{1}{l|}{0.001}   & \multicolumn{1}{l|}{0.00100}    \\ \hline
            \multicolumn{1}{|l|}{\textbf{02}}                  & \multicolumn{1}{l|}{7.99392}                                                       & \multicolumn{1}{l|}{-2.00004}           & \multicolumn{1}{l|}{1.001}              & \multicolumn{1}{l|}{1.00000}    & \multicolumn{1}{l|}{1.001}   & \multicolumn{1}{l|}{1.00000}    \\ \hline
            \multicolumn{1}{|l|}{\textbf{03}}                  & \multicolumn{1}{l|}{7.99390}                                                       & \multicolumn{1}{l|}{-0.00002}           & \multicolumn{1}{l|}{1.001}              & \multicolumn{1}{l|}{0.00000}    & \multicolumn{1}{l|}{1.001}   & \multicolumn{1}{l|}{0.00000}    \\ \hline
            \multicolumn{1}{|l|}{\textbf{04}}                  & \multicolumn{1}{l|}{7.99388}                                                       & \multicolumn{1}{l|}{-0.00002}           & \multicolumn{1}{l|}{1.00099}            & \multicolumn{1}{l|}{-0.00001}   & \multicolumn{1}{l|}{1.00099} & \multicolumn{1}{l|}{-0.00001}   \\ \hline
            \multicolumn{1}{|l|}{\textbf{05}}                  & \multicolumn{1}{l|}{7.99386}                                                       & \multicolumn{1}{l|}{-0.00002}           & \multicolumn{1}{l|}{1.00099}            & \multicolumn{1}{l|}{0.00000}    & \multicolumn{1}{l|}{1.00099} & \multicolumn{1}{l|}{0.00000}    \\ \hline
            \multicolumn{1}{|l|}{\textbf{06}}                  & \multicolumn{1}{l|}{7.99385}                                                       & \multicolumn{1}{l|}{-0.00001}           & \multicolumn{1}{l|}{1.00098}            & \multicolumn{1}{l|}{-0.00001}   & \multicolumn{1}{l|}{1.00099} & \multicolumn{1}{l|}{0.00000}    \\ \hline
            \multicolumn{1}{|l|}{\textbf{07}}                  & \multicolumn{1}{l|}{7.99384}                                                       & \multicolumn{1}{l|}{-0.00001}           & \multicolumn{1}{l|}{1.00098}            & \multicolumn{1}{l|}{0.00000}    & \multicolumn{1}{l|}{1.00098} & \multicolumn{1}{l|}{-0.00001}   \\ \hline
                                                               &                                                                                    &                                         &                                         &                                 &                              &
            %--------------------------------------------------------------------------- GAS USED --------------------------------------------------------------------------------------------------------
            \\
            \cline{2-7}
            \multicolumn{1}{c|}{}                              & \multicolumn{6}{l|}{\textbf{Transactions}}                                                                                                                                                                                                                                \\ \cline{2-7}
            \multicolumn{1}{l|}{}                              & \multicolumn{6}{l|}{00 – New service-account created}                                                                                                                                                                                                                     \\ \cline{2-7}
            \multicolumn{1}{l|}{}                              & \multicolumn{6}{l|}{01 – Emulator test accounts created (2)}                                                                                                                                                                                                              \\ \cline{2-7}
            \multicolumn{1}{l|}{}                              & \multicolumn{6}{l|}{02 – Emulator test accounts funded with 1.0 FLOW}                                                                                                                                                                                                     \\ \cline{2-7}
            \multicolumn{1}{l|}{}                              & \multicolumn{6}{l|}{03 – Deploy ExampleNFTContract into emulator}                                                                                                                                                                                                         \\ \cline{2-7}
            \multicolumn{1}{l|}{}                              & \multicolumn{6}{l|}{04 – Create a NonFungibleToken.Collection in each account}                                                                                                                                                                                            \\ \cline{2-7}
            \multicolumn{1}{l|}{}                              & \multicolumn{6}{l|}{05 – Mint an ExampleNFTContract.NFT into account01 Collection}                                                                                                                                                                                        \\ \cline{2-7}
            \multicolumn{1}{l|}{}                              & \multicolumn{6}{l|}{06 – Transfer ExampleNFT from account01 to account02}                                                                                                                                                                                                 \\ \cline{2-7}
            \multicolumn{1}{l|}{}                              & \multicolumn{6}{l|}{07 – Burn the ExampleNFTs from account02}                                                                                                                                                                                                             \\ \cline{2-7}
        \end{tabular}%
    }
\end{table}

\subsubsection{Analysis}
We included the overhead operations from Flow in this analysis and even with it, Flow is substantially cheaper to operate than Ethereum. The first conclusion from Table \ref{tab:FLOW_balance} is that most transactions cost the same value of 0.00001 FLOW (less than 0.00001€ at the time of this writing). This is the minimum fee in Flow \cite{flow2024d}, which may mean that the operation may have been even cheaper. Flow's fee system is based in three fee factors: an \textit{inclusion fee (IncFee)} to pay for the process of including the transaction into a block, transporting information, and validating signatures in the network; an \textit{execution fee (ExecFee)} to pay for FVM computations and operating on data storage, and a \textit{Surge fee (SrFee)} applied dynamically to modulate network usage and avoid surges. The total transaction cost can be calculated with: $ TxCost = (ExecFee + IncFee) \times SrFee $.
\par
Currently, the inclusion fee is fixed to 0.000001 FLOW, thus less than the minimum fee, and the surge factor is planned but not yet implemented. This means that, currently, transaction fees in Flow are mostly influenced by the computation effort required. But Table \ref{tab:FLOW_balance} shows that these do not exceed the minimum fee per transaction established in most cases. The most costly operations are the contract deployment and the new accounts creation, the latter substantially larger. The deployment transaction cost is proportional to the size of the contract in question. The contract used in this exercise occupies 6730 bytes of storage. Flow's documentation does not indicates the exact cost of storage per unit of memory consumed, but it estimates the cost of a deployment of a 50 KByte contract to be 0.00002965 FLOW. Our contract is substantially small, therefore the fee requested is coherent with the logic so far. The documentation also confirms that account creation is indeed the most expensive of all of Flow's base operations, but even in this case, a large part of that cost derives from the requirement for a minimum balance of 0.001 FLOW for accounts. This balance is included in the total cost of the transaction, though it is actually a transfer of funds between the account paying for the account creation process, and the new account. Without that value, the account creation process is actually twice the minimum transaction fee, i.e., 0.00002 FLOW (the cost indicated in Table \ref{tab:FLOW_balance} refers to the creation of two accounts).
\par
The comparison between the experiments from a cost perspective is quite extreme. The Solidity exercise required less transactions and yet it totalled almost 72 €. Flow required a series of extra operations to bring the system up to par, namely, to create a pair of extra accounts and respective collection resources, but even so, the total amount required only 2.00216 FLOW, which amounts to 1.72 €, a reduction by a factor of 42 compared to Ethereum. The Flow cost is actually a very conservative estimation of sorts. It includes 2.0 FLOW that were transferred to the two extra accounts to increase their balance above the minimum of 0.001 FLOW required to sustain the minimum of storage space and pay for transactions, which could be a much lower value and still allow the exercise to complete. On a cost basis, Flow is clearly the preferable option.

\subsection{Storage Comparison}
Storage is a critical aspect to consider in blockchain applications. Writing data into a blockchain block is an expensive operation because that data is going to be replicated in network nodes. This consideration has pushed blockchain to develop efficient data manipulation methods to optimise blockchain operation. This aspect is visible in smart contract development, whose code tends to be small and very optimised, when compared to other, non-distributed, applications. Another blockchain aspect that derives from this restriction is the complexity of the process of writing data into a block. Typically, this requires a digitally signed transaction and a gas cost paid upfront.
\par
Considering the importance of storing data in the blockchain, particularly to NFTs that use this data (metadata) to establish their uniqueness, we run a similar analysis to the one depicted in Sec. \ref{sec:cost_comparison} but towards determining the storage space considerations for the project.

\subsubsection{Storage in Ethereum}
Ethereum stores data in a \textit{contract-based} approach. The means that all data related to a given contract is stored \textit{referenced} to the address where the contract is deployed. Ethereum defines an "astronomically large array" indexed from the deployed contract address as the storage space of the contract. The array has $ 2^{256} $ potential slots, which is a number close to the number of atoms in the visible universe, hence the "astronomical" adjective, and each index in the array can store up to 32 Bytes \cite{Marx2018}. The "potential" derives from Ethereum only storing non-zero values, i.e., if a contract parameter has value 0, it does not count to the total storage used by the contract. Contract parameters are stored sequentially from index 0 but mappings, due to their dynamical nature, distribute their data throughout the storage space. Mappings use a key-value scheme and the index of a mapped element in storage by the hash of the key concatenated with a positional argument. This means that if a contract contains mappings or dynamic arrays, the data stored under its address is actually spread around the storage space. Which also complicates the storage analysis since Ethereum nor Solidity provides a direct way to determine the storage used, so we had to be creative and devise a method of our own for that effect. Following the logic used by Ethereum to store data, we wrote a script that checks storage slots for data, i.e., non-zero values. Table \ref{tab:space_calculations_ethereum} presents our findings.

\begin{table}[ht]
    \centering
    \caption{Storage analysis of the Solidity NFT contract}
    \label{tab:space_calculations_ethereum}
    \resizebox{\columnwidth}{!}{%
        \begin{tabular}{ccccc}
            \hline
            \multicolumn{1}{|c|}{\multirow{3}{*}{\textbf{Tx}}} & \multicolumn{4}{c|}{\textbf{Storage used by the ExampleNFT Solidity contract}}                                                                                                                                      \\ \cline{2-5}
            \multicolumn{1}{|c|}{}                             & \multicolumn{4}{c|}{0x5FbDB2315678afecb367f032d93F642f64180aa3}                                                                                                                                                     \\ \cline{2-5}
            \multicolumn{1}{|c|}{}                             & \multicolumn{1}{c|}{\textbf{Contract Size}}                                    & \multicolumn{1}{c|}{\textbf{Simple Storage}} & \multicolumn{1}{c|}{\textbf{Mapping Storage}} & \multicolumn{1}{c|}{\textbf{Total}} \\ \hline
            \multicolumn{1}{|c|}{00}                           & \multicolumn{1}{c|}{\multirow{4}{*}{10458}}                                    & \multicolumn{1}{c|}{96}                      & \multicolumn{1}{c|}{0}                        & \multicolumn{1}{c|}{10554}          \\ \cline{1-1} \cline{3-5}
            \multicolumn{1}{|c|}{01}                           & \multicolumn{1}{c|}{}                                                          & \multicolumn{1}{c|}{128}                     & \multicolumn{1}{c|}{94}                       & \multicolumn{1}{c|}{10680}          \\ \cline{1-1} \cline{3-5}
            \multicolumn{1}{|c|}{02}                           & \multicolumn{1}{c|}{}                                                          & \multicolumn{1}{c|}{128}                     & \multicolumn{1}{c|}{94}                       & \multicolumn{1}{c|}{10680}          \\ \cline{1-1} \cline{3-5}
            \multicolumn{1}{|c|}{03}                           & \multicolumn{1}{c|}{}                                                          & \multicolumn{1}{c|}{128}                     & \multicolumn{1}{c|}{0}                        & \multicolumn{1}{c|}{10586}          \\ \hline
            \multicolumn{1}{l}{}                               & \multicolumn{1}{l}{}                                                           & \multicolumn{1}{l}{}                         & \multicolumn{1}{l}{}                          & \multicolumn{1}{l}{}                \\ \hline
            \multicolumn{5}{|l|}{Transactions}                                                                                                                                                                                                                                       \\ \hline
            \multicolumn{5}{|l|}{00 - Deploy ExampleNFT Contract into the emulator}                                                                                                                                                                                                  \\ \hline
            \multicolumn{5}{|l|}{01 - Create an ExampleNFT into account01}                                                                                                                                                                                                           \\ \hline
            \multicolumn{5}{|l|}{02 - Transfer ExampleNFT from account01 to account02}                                                                                                                                                                                               \\ \hline
            \multicolumn{5}{|l|}{03 - Burn the ExampleNFT from account02}                                                                                                                                                                                                            \\ \hline
        \end{tabular}%
    }
\end{table}

The lion share of used storage is clearly the deployed contract. The Hardhat framework used in this case actually simplifies this one step by providing a \textit{size-contracts} feature that return the storage occupied by each deployed contract. The remaining values were found by running a storage scanning script to determine how many slots are in use and the amount of data stored in each. The \textit{Simple Storage} column refers to contract parameters such as \textit{totalSupply}, \textit{nextTokenId}, etc, while the \textit{Mapping Storage} refers to values stored in Solidity mappings specifically, which are used to establish the ownership logic in NFT contracts. Solidity establishes the existence of a NFT as a set of related mapping entries. All mappings in the ExampleNFT contract considered are only used for this purpose, which is confirmed in Table \ref{tab:space_calculations_ethereum}. The NFT only "exists" in transactions 01 and 02 and that is reflected in the storage used in mappings. As soon as the token is destroyed (burned) this data is deleted.

\subsubsection{Storage in Flow}

Compared to Ethereum, Flow simplifies this step greatly. It does not provide a tool per se, but determining the total storage used by an account is a trivial operation in Flow. As such, we developed a simple script to return the storage used by each account involved in the exercise and run it after each step. Table \ref{tab:FLOW_storage} presents our findings.

\begin{table}[ht]
    \centering
    \caption{FLOW token balance of each account in the exercise}
    \vspace{0.1cm}
    \label{tab:FLOW_storage}
    \resizebox{\columnwidth}{!}{%
        \begin{tabular}{lllllll}
            %--------------------------------------------------------------------------- GAS USED --------------------------------------------------------------------------------------------------------

            \hline
            \multicolumn{7}{|c|}{\textbf{Storage used by Flow accounts (Bytes)}}                                                                                                                                                                                                                                                                   \\ \hline
            \multicolumn{1}{|c|}{\multirow{2}{*}{\textbf{Tx}}} & \multicolumn{2}{c|}{\textbf{Emulator-account}}                                     & \multicolumn{2}{c|}{\textbf{account01}} & \multicolumn{2}{c|}{\textbf{account02}}                                                                                                            \\ \cline{2-7}
            \multicolumn{1}{|c|}{}                             & \multicolumn{1}{l|}{Storage (Bytes)}                                               & \multicolumn{1}{l|}{Difference}         & \multicolumn{1}{l|}{Storage (Bytes)}    & \multicolumn{1}{l|}{Difference} & \multicolumn{1}{l|}{Storage (Bytes)} & \multicolumn{1}{l|}{Difference} \\ \hline \hline
            \multicolumn{1}{|l|}{\textbf{00}}                  & \multicolumn{1}{l|}{428655}                                                        & \multicolumn{1}{l|}{}                   & \multicolumn{1}{l|}{0}                  & \multicolumn{1}{l|}{}           & \multicolumn{1}{l|}{0}               & \multicolumn{1}{l|}{}           \\ \hline
            \multicolumn{1}{|l|}{\textbf{01}}                  & \multicolumn{1}{l|}{429290}                                                        & \multicolumn{1}{l|}{635}                & \multicolumn{1}{l|}{1007}               & \multicolumn{1}{l|}{1007}       & \multicolumn{1}{l|}{1007}            & \multicolumn{1}{l|}{1007}       \\ \hline
            \multicolumn{1}{|l|}{\textbf{02}}                  & \multicolumn{1}{l|}{429607}                                                        & \multicolumn{1}{l|}{317}                & \multicolumn{1}{l|}{1007}               & \multicolumn{1}{l|}{0}          & \multicolumn{1}{l|}{1007}            & \multicolumn{1}{l|}{0}          \\ \hline
            \multicolumn{1}{|l|}{\textbf{03}}                  & \multicolumn{1}{l|}{436337}                                                        & \multicolumn{1}{l|}{6730}               & \multicolumn{1}{l|}{1007}               & \multicolumn{1}{l|}{0}          & \multicolumn{1}{l|}{1007}            & \multicolumn{1}{l|}{0}          \\ \hline
            \multicolumn{1}{|l|}{\textbf{04}}                  & \multicolumn{1}{l|}{436613}                                                        & \multicolumn{1}{l|}{276}                & \multicolumn{1}{l|}{1588}               & \multicolumn{1}{l|}{581}        & \multicolumn{1}{l|}{1588}            & \multicolumn{1}{l|}{581}        \\ \hline
            \multicolumn{1}{|l|}{\textbf{05}}                  & \multicolumn{1}{l|}{436769}                                                        & \multicolumn{1}{l|}{156}                & \multicolumn{1}{l|}{1751}               & \multicolumn{1}{l|}{163}        & \multicolumn{1}{l|}{1588}            & \multicolumn{1}{l|}{0}          \\ \hline
            \multicolumn{1}{|l|}{\textbf{06}}                  & \multicolumn{1}{l|}{436894}                                                        & \multicolumn{1}{l|}{125}                & \multicolumn{1}{l|}{1588}               & \multicolumn{1}{l|}{-163}       & \multicolumn{1}{l|}{1751}            & \multicolumn{1}{l|}{163}        \\ \hline
            \multicolumn{1}{|l|}{\textbf{07}}                  & \multicolumn{1}{l|}{437035}                                                        & \multicolumn{1}{l|}{141}                & \multicolumn{1}{l|}{1588}               & \multicolumn{1}{l|}{0}          & \multicolumn{1}{l|}{1588}            & \multicolumn{1}{l|}{-163}       \\ \hline
                                                               &                                                                                    &                                         &                                         &                                 &                                      &
            %--------------------------------------------------------------------------- GAS USED --------------------------------------------------------------------------------------------------------
            \\
            \cline{2-7}
            \multicolumn{1}{c|}{}                              & \multicolumn{6}{l|}{\textbf{Transactions}}                                                                                                                                                                                                                                        \\ \cline{2-7}
            \multicolumn{1}{l|}{}                              & \multicolumn{6}{l|}{00 – New service-account created}                                                                                                                                                                                                                             \\ \cline{2-7}
            \multicolumn{1}{l|}{}                              & \multicolumn{6}{l|}{01 – Emulator test accounts created (2)}                                                                                                                                                                                                                      \\ \cline{2-7}
            \multicolumn{1}{l|}{}                              & \multicolumn{6}{l|}{02 – Emulator test accounts funded with 1.0 FLOW}                                                                                                                                                                                                             \\ \cline{2-7}
            \multicolumn{1}{l|}{}                              & \multicolumn{6}{l|}{03 – Deploy ExampleNFTContract into emulator}                                                                                                                                                                                                                 \\ \cline{2-7}
            \multicolumn{1}{l|}{}                              & \multicolumn{6}{l|}{04 – Create a NonFungibleToken.Collection in each account}                                                                                                                                                                                                    \\ \cline{2-7}
            \multicolumn{1}{l|}{}                              & \multicolumn{6}{l|}{05 – Mint an ExampleNFTContract.NFT into account01 Collection}                                                                                                                                                                                                \\ \cline{2-7}
            \multicolumn{1}{l|}{}                              & \multicolumn{6}{l|}{06 – Transfer ExampleNFT from account01 to account02}                                                                                                                                                                                                         \\ \cline{2-7}
            \multicolumn{1}{l|}{}                              & \multicolumn{6}{l|}{07 – Burn the ExampleNFTs from account02}                                                                                                                                                                                                                     \\ \cline{2-7}
        \end{tabular}%
    }
\end{table}

The emulator-account displays an unusual large used storage but this is not relevant for our exercise. To speed up and simplify development, Flow provides its emulator environment with a series of standards already deployed into the emulator-account storage area, hence the unusual value at startup. We only make use of the \textit{NonFungibleToken} one, so most of that is not being used even.

\subsubsection{Analysis}
Flow presents itself as a more efficient alternative to save data in a blockchain. Ethereum saves data in 32-byte chunks arranged sequentially from the deployed contract address. To optimise storage, when possible, Ethereum saves two values into the same slot, if both are 16 bytes or less, by splitting one 32-byte slot into two of 16, but overall, apart from contracts, Ethereum saves data "discretely", i.e., in 32 or 16-byte chunks, except for strings, which are saved in UTF-8 using 2 bytes per character. Flow on the other hand presents a more granular storage, where data is discriminated to the byte.
\par
Regarding the size of the NFT construct itself, Ethereum is the better option in the sense that it produces a smaller data footprint. It is important to note that the NFT created in Flow was blank, i.e., without any additional metadata other than the bare minimum required by the standard, while the version in Ethereum was created with the recipient account address as metadata, stored as a UTF-8 string, and even so, Ethereum NFT was computed at 94 bytes of storage, while a programmatically smaller NFT from Flow required 163 bytes. Flow has a significantly more storage overhead than Ethereum, as in the data required for minimal functionality, but it still maintains a higher transactional throughput nonetheless. Though Flow does require more storage in absolute values, but price for storage in Flow, both in storage and transaction fees, is much less than in Ethereum, which makes this apparent disadvantage meaningless.
\par
Overall, Flow is still the better option in this case. Storage in Ethereum is a thorny issue in great part because of how expensive ETH has become as well. Also, this chain has a slower block rate than Flow, which means a lower rate of data writes, since these require transactions finalised in blocks. Flow does write more data to achieve similar functionality, but does so in a inexpensive and faster fashion, which balances the preference to its side.

\end{document}