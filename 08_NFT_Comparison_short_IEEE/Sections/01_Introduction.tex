\documentclass[../NFTComp_IEEE.tex]{subfiles}
\graphicspath{{\subfix{../Images}}}

\begin{document}
\section{Introduction}
\label{sec:introduction}
The inspiration for Flow resulted from an experience from its creators, \textit{Dapper Labs}, with the \textit{CryptoKitties}, one of Ethereum's first NFT projects. This project extended the NFT concept with a new usability layer absent from other similar projects. The contract minted a \textit{CryptoKitty}, a NFT representing a digital cat-like creature, and each kitty token was characterised by a unique genome parameter, an internal 256-byte string from which the metadata of the token were derived from. The token metadata was then fed into an image generator that showed the kitty's characteristics. Parameters such as eye color, skin color, ear type, etc were encoded in portions of the genome string. The innovative aspect of this project was that two CryptoKitties could be "bred" to generate a new one with a genome string that derived from the parent's genome. Dapper Labs established the genome mechanics such that new traits were acquired somewhat randomly and some traits were rarer than others. This new approach translated into a peak of popularity and a surge in transactions submitted in Ethereum, and that end up exposing the scalability and throughput limitations of Ethereum \cite{bbc2017}.
\par
Dapper Labs initially tried to solve these issues from within the Ethereum blockchain, but at some point it became clear that the blockchain needed architectural modifications to be able to overcome these throughput limitations. As such, instead of trying to "fix" Ethereum, Dapper Labs launched Flow in 2020 instead \cite{Gharegozlou2019}, a new blockchain solution developed from scratch centered around supporting NFTs and related mechanics. Flow presents several key differences from Ethereum, from how nodes behave in the network, the consensus algorithm used, to how data is stored and accessed in the chain, as well as similarities, such as defining and using a native cryptocurrency token to regulate blockchain operations (\textit{gas} in Ethereum) and smart contract support. Yet, Flow claims to present the same level of NFT functionalities as Ethereum and other similar NFT ready blockchains, but approaching the concept from a fundamental different point.
\par
This article introduces the Flow blockchain and how it organises itself from the architecture standpoint, followed by the introduction and analysis of a pair of simple implementations of a NFT minter smart contract, one in Cadence, Flow's smart contract programming language, and another in Solidity, Ethereum's equivalent. Then we compare both architectures and implementations towards determining the merits of Flow's claims as a viable and optimised alternative from NFT-based projects.
\par
The rest of this article is structured as follows: Section \ref{sec:related_works} overviews the publications relevant to this work. Section \ref{sec:flow_blockchain} provides an introduction to Flow, the blockchain central to this exercise. Section \ref{sec:architecture_comparison} compares implementations of NFT smart contracts in both Cadence and Solidity, as well as the supporting blockchain architectures, namely Flow and Ethereum. This article concludes with Section \ref{sec:conclusion}.
\end{document}