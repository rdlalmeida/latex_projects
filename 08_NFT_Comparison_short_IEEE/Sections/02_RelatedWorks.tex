\documentclass[../NFTComp_IEEE.tex]{subfiles}
\graphicspath{{\subfix{../Images}}}

\begin{document}
\section{Related Works}
\label{sec:related_works}
Academic research using Non-Fungible Tokens is as recent as the concept itself. The Quantum project that produced the first NFT happened a decade ago, therefore NFT-based research is necessarily younger. But even considering such small temporal window, the research community did produce a significant number of relevant publication that used NFTs in some capacity. The authors in \cite{Hung2023}, \cite{Barbuta2024}, and \cite{Sharma2024} explore a tokenisation approach to manage real estate, where houses, apartments, land plots, etc. are abstracted by NFTs since the mechanics used to operate with NFTs in a blockchain are quite similar to how real estate markets work, and real estate properties share the uniqueness and individuality of NFTs as well. A similar approach is followed by \cite{Chiacchio2022} where they use NFTs to abstract pharmaceutical products and use a \textit{digital twin} approach to ensure the traceability of a given product by mirroring the lifecycle of a NFTs within a blockchain with a series of checkpoints, as the product goes from the production line to where it is going to be distributed. This tokenisation trend continues with \cite{Karandikar2021}, where a similar strategy is used to propose a energy management system for microgeneration cases. The authors developed a blockchain-based environment where NFTs abstract actors in the system, i.e., solar panels, battery packs, wind turbines, consumers, utility companies, etc. and the values exchanged in the system, i.e., electric energy and money, are abstracted with cryptocurrencies. Another example using the same strategy is found in \cite{Regner2019} where an event management system uses NFTs to abstract event tickets, taking advantage of the same uniqueness and individual elements, such as allocated seat, event name, id number, etc, that characterise these tickets and how they can be encoded into the metadata of an NFT.
\par
Other opted for a higher lever approach and presented an analysis based on the architectural aspects of NFTs rather than use them as a simple building block in a solution. \cite{Hong2019} explores the architectural aspects of specific NFT implementations in the Hyperledger environment, a framework to develop private custom blockchains. \cite{Yang2022} and \cite{Bal2019} present a similar high level architectural approach but with a specific scope in mind, namely, tracing and value transfer applications. Even though NFT is a recent technology, \cite{Wang2021b} and \cite{Ma2023} presented \textit{systematisation of knowledge (SoK)} articles about this technology, but they did not approached any architectural aspects of the technology. All publication mentioned thus far used Ethereum and Hyperledger for the examples and prototypes developed, and none even mentioned Flow as an alternative. In that regard, to the time of this writing, we found no academic publications using Flow as a development platform, let alone exploring if the new architectural approach could benefit their solutions. The few mentions to Flow in academic examples came from review style papers, namely \cite{Wang2021}, \cite{Razi2024}, and \cite{Guidi2023}, the latter providing the most extensive explanation.

\subsection{Our Contribution}
This paper presents a detailed exploration of the Flow blockchain as an architectural alternative to implement NFTs. To illustrate the differences, we also use Ethereum as an example of a general-purpose blockchain for comparison. We also present a concrete example of a simple NFT contract using Cadence, the smart contract programming language used in Flow, and compare it to a Solidity version of a functionally equivalent NFT contract. This exercise finalises with the analysis of the results obtained.

\end{document}