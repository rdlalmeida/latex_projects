\documentclass[../NFTComp_IEEE.tex]{subfiles}
\graphicspath{{\subfix{../Images}}}

\begin{document}
\section{Conclusion}
\label{sec:conclusion}
This paper presents an implementation analysis for two distinct architectures to create NFT-based contract capable of minting and transferring these tokens between users. Ethereum and Flow were chosen for this exercise due to former's role as a reference, general-purpose, and very popular blockchain, the latter being a blockchain created with the specific purpose of solving known issues with the Ethereum chain, namely, by supporting high throughput and scalable NFT projects.
\par
Flow is quite younger than Ethereum and, because of this, is still under some development and missing a wider spectrum of applications when compared to a more mature Ethereum. Nevertheless, Flow provides more than enough capabilities to produce a NFT-capable contract. If anything, the structural analysis done in Sec. \ref{sec:flow_blockchain} reveals Flow as the more complex, but also more configurable blockchain. From a programmatic point of view, Cadence is syntactically more complex than Solidity and contracts consume more time to develop, but the deployment process is simpler and faster than the Ethereum equivalent. The cost aspect favors Flow as well, though the limitations of Ethereum in that aspect fall outside of the technological scope. ETH high price is a much more consequence of speculative action than from an intrinsic value as a requirement to execute transactions in the network. Flow's lack of relative popularity when compared with Ethereum works in its favor by providing a much cheaper and realistic alternative to NFT-based applications since FLOW's price is not yet affected, at least not as much, from the same speculative forces that risk making Ethereum simply too expensive to use for regular users.
\par
Flow presents a strong case for a much cheaper alternative to Ethereum for NFT-based applications. It is also provides a self-contained programming language for smart contracts in Cadence. Flow uses the same language to create smart contracts, scripts and transactions used to interact with the former. From a developer standpoint, it is a simpler approach than Ethereum, which uses Solidity for smart contract development only, but interacting with contracts deployed in the blockchain, which includes all NFT-related mechanics, requires an additional, third party framework, such as Hardhat, Truffle, Remix, etc.
\par
From a storage point of view, Flow is more complex to understand, but easier to use. The UNIX-like approach to storage paths makes it easier to understand due to its similarity with regular operating systems. Ethereum's storage model is simpler but hard to operate with, especially when complex structures such as mappings are stored. The contract-based approach is more transparent. Realistically, as long as one knows the address of the deployed contract, it is possible to read all data stored in the contract, if not encrypted. The process is not simple but it is doable. Flow's account-based storage approach presents a cleaner approach. Owning digital objects in Flow is a easier concept to understand because of the exclusivity of the storage area associated to an account. Only the owner can access this area, regardless of where the contract that created the digital resource is actually saved.
\par
Though Flow offers clear advantages, Ethereum still has the lion share of the NFT-based applications in the blockchain ecosystem. The 7 years that Ethereum has on Flow are a significant gap to overcome, especially considering that Ethereum was the only realistic alternative to smart contract development for a significant part of that period, which boosted its popularity to levels difficult to replicate. Also, Flow is still in relative development. Some of the concepts indicated in this article were introduced or updated significantly in the recent Crescendo upgrade \cite{flow2024f} from September 2024, such as entitlements, capabilities and access control mechanics. Even with all this considered, Flow does present a rich, albeit limited, application ecosystem mostly focused on digital collectibles, which is the type of applications that inspired the creation of this blockchain in the first place.
\end{document}