\documentclass[../NFTComp_IEEE.tex]{subfiles}
\graphicspath{{\subfix{../Images}}}

\begin{document}
\section{Conclusion}
\label{sec:conclusion}
This paper presents an implementation analysis for two distinct architectures used to create NFT-based contracts. Ethereum and Flow were chosen for this exercise due to former's role as a general-purpose and popular blockchain used as reference, the latter a specific blockchain created to solve known throughput and scalability issues with Ethereum.
\par
Despite being significantly younger than Ethereum, Flow provides enough features to produce NFT-capable contracts. The structural analysis in Sec. \ref{sec:flow_blockchain} reveals Flow as a more complex, but also more configurable blockchain. From a programmatic point of view, Cadence is syntactically more complex, contracts consume more time to develop, but the deployment process is simpler, faster and produce smaller deployable code than the Ethereum equivalent. The cost aspect favors Flow as well, though the limitations of Ethereum in that aspect fall outside of the technological scope considered. Flow's lack of relative popularity works in its favor by providing a cheaper and realistic alternative to NFT-based applications since FLOW's price is not as affected from the same speculative forces that risk making Ethereum too expensive to use.
\par
Though Flow offers clear advantages, Ethereum still has the lion share of the NFT-based applications in the blockchain ecosystem. The 7 years that Ethereum has on Flow are a significant gap to overcome. Ethereum was the only realistic option for smart contract development for a long time, which boosted its popularity to levels difficult to replicate. Nevertheless, Flow does present a rich, albeit limited, application ecosystem focused on digital collectibles, which is the type of applications suitable for a NFT intensive blockchain.
\end{document}