\documentclass[../NFTComp_IEEE.tex]{subfiles}
\graphicspath{{\subfix{../Images}}}

\begin{document}
\section{Introduction}
\label{sec:introduction}
The inspiration for Flow resulted from an experience from its creators, \textit{Dapper Labs}, with the \textit{CryptoKitties}, one of Ethereum's first NFT projects. This project extended the NFT concept with a new usability layer absent from similar projects. The contract minted a \textit{CryptoKitty}, an NFT representing a digital cat-like creature, and each kitty token was characterised by a unique genome parameter, an internal 256-byte string from which the metadata of the token was derived. Feeding this metadata into an image generator produces an image from the genome string, which is used to encode parameters such as eye and skin colour, ear type, etc. The innovative aspect of this project is that two CryptoKitties can be "bred" to generate a new one with a genome string that derives from the parent's genome. Dapper Labs established genome mechanics that produced traits randomly, which led to different rarity levels for individual kitties. This new approach translated into a peak of popularity and transactions submitted in Ethereum, which led to network stoppages stemming from scalability and throughput limitations of the network \cite{bbc2017}.
\par
Dapper Labs initially tried to solve these issues from within the Ethereum blockchain, but at some point it became clear that the blockchain needed architectural modifications to be able to overcome these throughput limitations. As such, instead of trying to "fix" Ethereum, Dapper Labs launched Flow in 2020 instead \cite{Gharegozlou2019}, a new blockchain solution developed from scratch centred around supporting NFTs and related mechanics. Flow introduces many innovations, among the most remarkable are the definition of a single abstraction for accounts, thus overcoming the distinction between EOA and contract accounts in Ethereum and the introduction of a new language, Cadence, that allows ownership to be defined at the language level through the resource construct.
Flow presents several furher key differences from Ethereum, from how nodes behave in the network, the consensus algorithm used, to how data is stored and accessed in the chain, as well as similarities, such as defining and using a native cryptocurrency token to regulate blockchain operations (\textit{gas} in Ethereum) and smart contract support. Flow uses a fundamentally different approach and claims to present the same level of NFT functionalities as Ethereum and other NFT-ready blockchains.
\par
This article introduces the Flow blockchain and its architecture, followed by the introduction and analysis of a pair of implementations of an NFT minter smart contract, one in Cadence, Flow's smart contract programming language, and another in Solidity, Ethereum's equivalent. Then we compare both architectures and implementations towards determining the merits of Flow's claims as a viable and optimised alternative to NFT-based projects.
\par
This article continues in Section \ref{sec:related_works} with an overview of the publications relevant to this work. Section \ref{sec:flow_blockchain} provides an introduction to Flow, the blockchain central to this exercise. Section \ref{sec:architecture_comparison} compares implementations of NFT smart contracts in both Cadence and Solidity, as well as the supporting blockchain architectures, namely Flow and Ethereum. This article concludes with Section \ref{sec:conclusion}.
\end{document}