\documentclass[../NFTComp_IEEE.tex]{subfiles}
\graphicspath{{\subfix{../Images}}}

\begin{document}
\section{Related Works}
\label{sec:related_works}
NFTs were introduced about a decade ago with the Quantum project \cite{Exmundo2023} and were popularised through Ethereum and the smart contract technology it introduced. Though NFTs existed in a small temporal window, the research community did produce a significant number of relevant publications that used them in some capacity. The authors in \cite{Hung2023}, \cite{Barbuta2024}, and \cite{Sharma2024} explore a tokenisation approach to manage real estate, where houses, apartments, land plots, etc. are abstracted by NFTs. Real estate properties are unique and individual, similar to NFTs, and the blockchain mechanics that regulate these tokens are very similar to how real estate markets operate. \cite{Chiacchio2022} follows a similar approach where they use NFTs to abstract pharmaceutical products, \cite{Karandikar2021} uses a similar strategy to propose an energy management system for microgeneration, and \cite{Regner2019} provides an example where an event management system uses NFTs to abstract event tickets, taking advantage of the same uniqueness and individual elements.
\par
Others opted for a higher-level approach and presented an analysis based on the architectural aspects of NFTs rather than using them as a simple building block in a solution. \cite{Hong2019} explores the architectural aspects of specific NFT implementations in the Hyperledger environment, a framework to develop private custom blockchains. \cite{Yang2022} and \cite{Bal2019} present a similar high-level architectural approach but for tracing and value transfer applications. \cite{Wang2021b} and \cite{Ma2023} presented \textit{systematisation of knowledge (SoK)} articles about this technology, but they did not approach any architectural aspects. All publications mentioned thus far used Ethereum and Hyperledger for the examples provided, and none even mentioned Flow as an alternative. In that regard, at the time of this writing, we found no academic publications using Flow as a development platform, let alone exploring if the new architectural approach could benefit their solutions. \cite{Wang2021}, \cite{Razi2024}, and \cite{Guidi2023} provided the few mentions of Flow in academic examples, with \cite{Guidi2023} providing the most extensive explanation of all.
\par
Flow abstracts digital objects through a novel \textit{Resource Oriented Paradigm (ROP)}, an approach similar to the popular object-oriented paradigm. Sec. \ref{sec:resource_oriented_paradigm} goes in deeper detail. ROP is a feature that differentiates Flow from Ethereum and, currently, we found only two public blockchains using this paradigm, namely the \textit{Aptos} \cite{Aptos2022} and \textit{Sui} \cite{Sui2023} blockchains. Both chains were written using \textit{Move}, a ROP-based smart contract programming language developed by \textit{Meta} (formerly known as \textit{Facebook}) back in 2019 \cite{daic2024}. \textit{Move} derives strongly from \textit{Rust}, just like \textit{Cadence}, the smart contract programming language developed by Flow, which makes these blockchains functionally very close. But Move, as well as the \textit{Movement} ecosystem that bases itself in this language \cite{Movement2024}, are very recent additions to the blockchain universe. \cite{Benetollo2023} presented the only academic publication based in a ROP language found.

\subsection{Our Contribution}
This paper presents a detailed exploration of the Flow blockchain as an architectural alternative to implement NFTs. To illustrate the differences, we also use Ethereum as an example of a general-purpose blockchain for comparison. We also present a concrete example of a simple NFT contract using Cadence, the smart contract programming language used in Flow, and compare it to a Solidity version of a functionally equivalent NFT contract. This exercise finalises with the analysis of the results obtained.
\end{document}