\documentclass[../main.tex]{subfiles}
\graphicspath{{\subfix{../../Images}}}

\begin{document}
\section{Research Plan}
\label{sec:research_plan}
The wave of decentralised innovation triggered by the introduction of Bitcoin and blockchain in 2009 promised to bring new solutions through the injection of a decentralised computation paradigm into old problems. Governance and electronic voting methods were among the scientific areas that were expected to be influenced by the new approach, as well as being where my research interests currently lie. As such, I enrolled in a doctoral program in the area towards being able to pursue these interests in an organised and structured fashion.
\par
My doctoral research process begun with an extensive literature survey around the topic, which revealed a clear correlation between the introduction of technological advances that are passible to be useful in the e-voting context, and how these would be reflected in academic proposals some time after. This tendency was verifiable from the development of the first commercial encryption schemes up to the introduction of smart contracts and the blockchain virtual machines that run them.
\par
All except for \textit{Non-Fungible Tokens (NFTs)}, one of the latest and most popular of blockchain features. NFTs provide a clever method to establish ownership relations in the blockchain using a secure and transparent mechanism. Pairing this with encryption techniques, NFTs also provide a private method to add data to a blockchain via their metadata properties, which make them particularly attractive to use as vote ballots in a decentralised e-voting proposal. Yet, up to this point, no academic proposals were submitted exploring this approach to e-voting, nor even exploring the implementable side of this constructs. NFTs were adopted very quickly by the research community and there are many examples of NFT-based academic projects, just not in the e-voting context presently.
\par
As such, I intend to explore this research gap by exploring and comparing existing NFT architectures and use the these to develop a NFT-based e-voting proposal that can be use to infer the usefulness of this feature in the overall theme of blockchain voting systems.
\vspace{3mm}
\textbf{Collaboration with the University of Surrey}
\vspace{3mm}
\par
This doctoral program establishes a period of 6 months minimum to a maximum of one year where the doctoral research must be undertaken in collaboration with a foreign (non-Italian) university or research institution. I chose the University of Surrey for this purpose due to a prior successful research collaboration. My advisor team in Pisa as also collaborated with Surrey in several activities in the past related with blockchain research, as well as my own very positive past experiences residing and working within the United Kingdom.
\par
If agreed, the expectation is to start this collaboration by February 2025. I'm planning to have a NFT architectural comparison article ready for submission by that time, so I expect to use the bulk of this period to continue and finalise the development of the NFT-based e-voting system, proceed to the experimental results phase. The expectation is to publish an article detailing this proposal, supported with experimental data as well.
\par
The idea is to make this collaboration opportunity as reciprocal as possible. During that period I'm also expecting to collaborate with any research opportunity from the faculty consistent with the main context of my research focus.
\end{document}