\documentclass[../main.tex]{subfiles}
\graphicspath{{\subfix{../Images}}}

\begin{document}
\section{Related Works}
\label{sec:related_works}
The commercialisation of cryptography triggered a stream of new approaches to e-voting systems using new cryptographic schemes and tools derived from these to implement these ideas. Diffie and Hellman's 1976 publication \cite{Diffie1976} was followed by new symmetrical and asymmetrical cryptographic schemes proposals, such as \cite{Chaum1981}, \cite{ElGamal1984}, and \cite{Rivest1983}, which in turn were instrumental to define cryptographic tools that have been extensively used in e-voting system development, such as blind signatures \cite{Chaum1983}, Mix-Nets \cite{Chaum1988}, Homomorphism in threshold cryptosystems \cite{Shamir1979} and cryptographic knowledge proofs \cite{Goldwasser1986}.
\par
Research in e-voting systems progressed towards the establishment of a classification criteria that were then used to compare proposals from a security standpoint. Authors implemented criteria such as \textit{accuracy}, \textit{privacy}, \textit{eligibility}, \textit{verifiability}, \textit{convenience}, \textit{flexibility}, \textit{mobility} and \textit{robustness} using the cryptographic tools indicated thus far. A simple example that illustrates this process is the usage of asymmetrical encryption keys to encrypt voter data, thus protecting the \textit{privacy} of the voter. A proposal that uses such scheme can claim that it establishes voter \textit{privacy}. Yet, a formal definition of such criteria has notorious absent from related literature. \cite{Neumann1993} was among the first to attempt such characterisation, with subsequent publications, such as \cite{Fujioka1992}, \cite{Baraani1995}, \cite{Juang1997}, \cite{Ku1999}, \cite{Lee2000}, \cite{Joaquim2003}, \cite{Baiardi2005}, and \cite{Chaum2007}, continuing this trend. These articles followed the rationale that the more security of a voting system is proportional to the number of security criteria it implements, which translates in an assurance that a voter can trust his/her choice to it. Over time, these cryptographic tools became a fixture in all e-voting proposals in this initial 30-year window of research in centralised e-voting systems. This is a limiting paradigm since it constrains the whole system by establishing a single point of attack or failure, while also reducing system scalability, due in great part to the demand of a large amount of resources, such as primary and secondary storage, computational power, network bandwidth, etc., to implement these criteria.
\par
Scalability is an crucial characteristic that can hinder a wide adoption of the proposed system. Contemporary elections can go from exercises where the system is only expected to process up to a thousand votes at one point, to national-wide events that might require the processing of millions of votes. The relationship between the scalability of a system and the amount of available resources is evident in the analysed literature. Proposals from this era confirm that the ones that satisfy the most security criteria also establish a computationally complex and demanding system that is often limited to small-scale elections. As an example, in \cite{Cranor2002}, \cite{Nurmi1991}, \cite{Iversen1992}, and \cite{Niemi1999}, this trade-off was shifted towards security and transparency at the expense of scalability. The authors do recognize this limitation and, as such, are explicit in restricting the usage of their system to small-scale elections.
\par
On the other side of this spectrum, proposals such as \cite{Benaloh1994}, \cite{Benaloh1986a}, \cite{Park1994}, \cite{Juang2002}, \cite{Okamoto1996}, \cite{Okamoto1998}, \cite{Magkos2001} or \cite{Moran2006} present scalable systems that are simpler than the ones considered in the last paragraph, but their adoption in large-scale elections implies a sacrifice in security and transparency. Proposals suitable for large-scale elections implement the lowest number of security criteria.
\par
Several e-voting systems were evaluated in a real-world scenario. For this case, we are only interested in systems that address the criterion of \textit{mobility}, i.e., a voting system that does not restrict voters geographically, in part because these proposals did not follow any of the academic ones that preceded them. Therefore, there was little interest in electronic proposals that limited their users to traditional polling places, since they infuse a degree of privacy and security that derives solely from the surrounding election apparatus. The few notable exercises in recent history were run in Canada in 2013 \cite{Goodman2014}, Estonia in 2005 \cite{Heiberg2005}, Switzerland in 2005 \cite{Binder2019}, Norway in 2011 \cite{Barrat2012}, France in 2012 \cite{Pinault2012}, and Australia in 2015 \cite{Halderman2015}.
\par
The first decentralised e-voting proposals were limited to cryptocurrency-centric blockchain, such as Bitcoin, due to the lack of alternatives in the early years of blockchain development. This proposals were somewhat simple in the sense that they conceived convoluted and impractical methods to exchange information using transactional metadata from cryptocurrency transfers.  Proposals such as \cite{Zhao2016}, \cite{Cruz2016}, \cite{Bistarelli2017}, \cite{Lee2017}, \cite{Shaheen2017}, \cite{Wu2017}, \cite{Dimitriou2020} or \cite{Bartolucci2018} used a script function available in Bitcoin transactions to add voting information to the blockchain data, namely the \textit{OP\_RETURN} function, which receives an 83-byte wide string as input, and adds it to the transaction metadata as the function's output. Hence, it was used to write non-transactional data directly to the blockchain. This mechanism needs to go around the limited functionalities offered by early blockchain solutions whose scope of operations were limited to cryptocurrency transactions. Furthermore, this method is infeasible for large-scale use because a Bitcoin transaction necessarily involves exchanges worth a considerable amount of money, as well as being notoriously hard-to-scale blockchain due to its low block rate. Bitcoin adds a new block every 10 minutes, which limits the rate of operations that this blockchain can withstand.
\par
The popularisation of public blockchains attracted interest from other platforms, which triggered the development of software frameworks used to create and deploy custom blockchains with proprietary access control and offering more flexibility to applications. As such, researchers such as \cite{Kirby2016}, \cite{BenAyed2017}, \cite{Chaieb2018}, \cite{Burhanuddin2018}, \cite{Zhang2018}, \cite{Khan2018}, \cite{Murtaza2019}, \cite{Faour2019}, \cite{Killer2020}, \cite{Han2020}, \cite{Vivek2020}, \cite{Mani2022}, \cite{Zhou2020}, \cite{Alvi2022}, \cite{Hassan2022}, \cite{Vidwans2022} or \cite{Matile2019} adopted private blockchains using custom-made solutions in customisable frameworks such as Hyperledger Fabric, Quorum, and Multichain. As a drawback, the increased flexibility is paid for by a lack of network support. It is difficult to establish a privately accessible network with enough active nodes to establish a satisfactory level of redundancy.
\par
A significant breakthrough arrived with the introduction of the smart contract through the Ethereum blockchain, specifically through the implementation of a \textit{Touring-complete} processing platform, named \textit{Ethereum Virtual Machine (EVM)}, that can execute code scripts in a decentralised fashion by splitting and distributing the instructions through the active nodes in the network. Proposals such as \cite{McCorry2017}, \cite{Koc2018}, \cite{Dagher2018}, \cite{Fusco2018}, \cite{Hjalmarsson2018}, and \cite{Mols2020} were among the first to implement a voting system via Ethereum smart contracts.
\par
The same time period also produced some blockchain-based proposals for e-voting systems in a real-world scenario. Unlike the centralised approach, these solutions have significant overlap with the academic proposals considered. Among these real-world examples, we cite \textit{Follow My Vote} \cite{Ernest2021}, \textit{TiVi} \cite{TiVi2021}, \textit{Agora} \cite{AgoraWhitePaper2021}, and \textit{Voatz} \cite{Voatz2021}. The difference between the nature of approaches regarding their real-world applications is an indicator of the potential of blockchain in this scenario. Real-world blockchain-based e-voting solutions follow the academic approach closer than their centralised counterparts.
\par
The real-world solutions indicated are end-to-end applications, i.e., they are ready to be used in an election, as long as their limitations are properly addressed (mostly related to scalability). But there are other proposals published in the public domain as protocols that can be used to set up an e-voting systems instead. These are not complete solutions, as the ones indicated thus far, but instead protocols that can be used to establish a secure and transparent e-voting system. \cite{Lamtzidis2023} presents a concise summary of the most relevant Ethereum based protocols in existence. Most of the logic employed in these protocols is already abstracted through smart contracts already deployed and publicly available in the Ethereum blockchain, such as the \textit{MACI (Minimal Anti-Collusion Infrastructure)} protocol \cite{MACI2024}, \textit{Semaphore} \cite{Semaphore2024}, \textit{Cicada}, and \textit{Plume}. It is important to notice that none of these protocols employs NFTs as an abstraction of votes as well. There are references to NFTs in the protocol description, but these are used to exemplify how the protocols handles ownership of digital objects or using NFT ownership as means to verify the identity of a voter, but never as the main vote element.
\par
So far, none of the proposals considered used, or even mentioned, NFTs in their processes. Nevertheless, a search for recent proposals with this characterizing element was conducted. Any usage of NFTs in any capacity in a remote voting system was considered relevant, yet our search was unable to find a single complete proposal that combined both. For this purpose, we consulted the main academic databases, namely \textit{Google Scholar}, \textit{Science Direct}, \textit{IEEE}, and \textit{ACM}, using a broader search term at first, namely, "e-voting" and "NFT," as well as with expanded acronyms and other variations, without success. The closest article to a NFT-based e-voting system we were able to consider was \cite{Sagar2023}, where the authors use SoulBound NFTs, a special case of non-transferable NFTs that can potentially be used for identification purposes \cite{Weyl2022}, to circumvent the need for a trusted third party to implement voter \textit{eligibility}. The proposed system only interacts with these NFTs during voter validation. The article does not provide enough technical details to determine exactly how blockchain is used in the remainder of the voting process, but it is clear with how limited their use of NFTs is.
\par
Two other proposals, namely \cite{Bistarelli2022} and \cite{Agbesi2019}, do mention Non-Fungible Tokens, but more so as a product of their literature and technology review and not as an integral component in their solution. To conclude this search process, \cite{Ali2023}, \cite{Bao2022}, and \cite{Wang2021} produced extensive surveys around the potentials and usage of NFT, as well as providing a list of future challenges where this technology can be determinant. \cite{Ali2023} does mention a potential application of NFTs in governance applications, but without specifying voting or even elections in any capacity. \cite{Wang2021} listed challenges limited to purely digital applications, namely gaming, virtual events, digital collectibles, and metaverse applications. \cite{Bao2022} provided a broader survey and identified a larger and more specified set of potential applications for NFTs, but none of them related to governance or e-voting.
\par
As far as we were able to determine, no proposals were submitted thus far using NFTs as an integral element of an e-voting system.
\end{document}