\documentclass[../main.tex]{subfiles}
\graphicspath{{\subfix{../../Images}}}

\begin{document}
\section{Introduction to Non-Fungible Tokens}
\label{sec:introduction_nfts}

\textit{Non-Fungible Tokens (NFTs)} follow cryptocurrencies as another base feature made possible by blockchains. From the network point of view, cryptocurrencies associate user addresses to a value: the cryptocurrency balance of that account, in that network. NFTs invert this logic to create a new way to represent digital ownership. Each NFT is created as digital object but with an internal parameter (typically named 'id' or similar) that ensures that every object is unique by setting them with a different id. From the point of view of the blockchain, a NFT is a series of synchronized records that create an unequivocal relationship between the unique id of the NFT and the account address that owns it.
\par
The NFT concept is transversal to all blockchains, but since the concept was introduced through the Ethereum blockchain, the NFT standard is regulated by two \textit{Ethereum Improvement Proposals (EIP)}, namely EIP-721 and EIP-1155. These proposals define a set of base characteristics (variables and functions) that a smart contract needs to implement to conform to the standard. These requirements are abstracted in the respective \textit{Ethereum Request for Comments (ERC)} standards, ERC-721 \cite{ERC721} and ERC-1155 \cite{ERC1155}. It is possible for someone to define an NFT outside of these standards since they are not legally enforceable. Yet, since the inception of this concept, the vast majority of published NFTs follow this standard since this gives them a level of default interoperability that is hard to achieve otherwise. If a given NFT implements the standards indicated, other users and developers have the assurance, due to the function and parameter requirements that are "forcibly" implemented through the usage of these standards, that the variables and functions defined in the interface are implemented in the NFT contract, similar to what already happens with interfaces in object-oriented programming paradigms. For example, if a user is interacting with a NFT smart contract that implements the ERC-721 standard, the user can call the function 'balanceOf' without needed to check the smart contract code to see if it was implemented.The user can provide an address as argument to it and find out how many NFTs from that contract the given address currently owns. The 'balanceOf' function is made available by the interface and it is available by any contract that implements the ERC-721 interface. Some interface functions can be overwritten to include additional functionalities or requirements, but this process preserves the function name and, therefore, the guarantee that if a user invokes it without checking the contract code a priori, this invocation does not raise any errors and retains the basic functionalities.
\par
The application potential of this technology regarding digital collectibles has in itself motivated the creation of NFT-centric blockchains, such as Flow \cite{Hentschel2019a}, which were developed towards overcoming aspects that make NFT mechanics too expensive, both in gas spent, resources allocated and execution time, in more general purpose blockchains such as Ethereum for example, as well as online marketplaces dedicated solely to the commercialization of NFTs (e.g., OpenSea \cite{OpenSea2024}) as long as the NFT smart contract implements the standards indicated. Popular NFT marketplaces usually only accept tokens whose implementing smart contracts ensure the implementation of some standards, ERC-721 one of the fundamental ones.
\par
Unlike cryptocurrencies, NFTs can store metadata onchain. Yet, because of the high cost associated with writing operations, in most cases, to optimise cost, most of the metadata stored in a NFT is a URL that can be automatically resolved to an off-chain resource, typically an image, a video, or any other type of digital file. This is the most common approach with artistic NFTs and even most digital collectibles \cite{Trautman2022}.

\subsection{NFTs vs. Cryptocurrencies}
In a blockchain context, NFTs represent objects, while cryptocurrencies are variables. With cryptocurrencies, account addresses are the "keys" while their cryptocurrency balance is used as "value", if a key-value scheme is used to represent the cryptocurrency balances in the network. Account addresses have their balances being modified through transactions, but the correspondence in this case is always 1:1, i.e., one account address to one cryptocurrency balance. When working with NFTs, this relationship is also 1:1, but the arguments are inverted, with the NFT's ids as the unique "keys" and the owner's address as the corresponding "value" in the same key-value scheme. This arrangement reinforces the unique nature of NFTs by enforcing the uniqueness of their id set, i.e., \textit{digital scarcity}, while their respective owners can change, thus allowing NFTs to be transferred among accounts. This aspect is also what gives NFTs their "Non-Fungible" characteristic. Since each has their own record, each NFT is it's own unique digital record and cannot be exchanged by another with the same characteristics because, by definition, there is none. Cryptocurrencies work inversely, and their tokens do not have any kind of unique identifiers, therefore any unit of a cryptocurrency is functionally equivalent to all other units of the same token. Users can exchange equal quantities of a cryptocurrency without losing anything in the process (other than transaction fees), thus exploiting their fungible aspect.
\par
Though most NFTs produced thus far represent \textbf{digital} objects, there are no strict requirements in that regard. NFTs can be used to represent physical object, or more correctly, to establish ownership of that physical object in a digital distributed ledger, but so far the emphasis has been in keeping everything contained in the digital realm. The amount of cryptocurrency owned by an account is determined by either a balance value stored in the governing contract, which is the case for the majority of ERC-20 (the standard that regulates cryptocurrency contracts in the Ethereum network) tokens, or by determining the Unspent Transactional Output (UTxO) value associated with the account, which is the strategy used by Bitcoin and a few other cases. Adding data to a blockchain using only cryptocurrencies is a challenge in itself, since these do not provide a direct mechanism to write arbitrary data into a block. Before the NFT standard, researchers looking to use a cryptocurrency-based blockchain for alternative applications had to be creative in that regard. One of the more popular approaches is using transactional metadata to add extra data to the transaction that gets written into a block.
\par
Bitcoin was extensively explored in this regard, since it was the sole blockchain in operation until Ethereum was introduced in 2015. Bitcoin added new functionalities somewhat periodically, and its $ 0.9.0 $ version introduced the \textit{OP\_RETURN} instruction to its execution set. When run, this instruction always writes its input, unchanged, into the transactional data that gets written into the blockchain \cite{Bartoletti2017}. This provided researchers with an alternative to sending arbitrary data to a blockchain that was not conceived with this functionality in mind.
\par
NFTs and smart contracts provide more efficient and flexible means to achieve the same result. When an NFT gets minted, its metadata is written into a block in the chain. Depending on the actual implementation, this data can be changed afterwards but only through a transaction digitally signed by the NFT owner.

\subsection{Contract-based and Account-based blockchains}
\label{sec:contract-to-account-storage}
For contract-based blockchains such as Ethereum, NFT metadata is always stored relative to the contract that implements the NFT itself, with information indicating which user "owns" it (typically through an address-to-NFT-identifier mapping). For account-based blockchains such as Flow, NFTs are always and uniquely stored in an account-based location, which can be the minting contract account (address to where the NFT contract was deployed), another contract, or a user account.
\par
In a contract-based blockchain, the information of every NFT minted by a contract is always visible in the deployed contract code, i.e., by checking the deployment address of that contract, which leads to the block where the NFT contract is written, it is possible to check the NFT details or metadata for all NFTs minted by that contract, even the ones that were "burned." This means that if the contract gets destroyed, all NFT ownership information is lost permanently. For example, in Ethereum, Solidity contracts can be programmed with a self-destruct function that ensures this result. The substantial differences between these architectures justifies an exploration on how NFTs might fare in a e-voting context.

\subsection{Token Burning}
Token ownership in a blockchain is abstracted by the ability of a user to access the account containing that token or the capability of that user to generate a valid digital signature that can sign a transaction to transfer that token to some other location. This implies that the user owns or controls the private encryption key required to generate this signature. Deriving an account address from a private encryption key is a trivial operation, but the opposite is computationally infeasible. The ownership mechanism of a blockchain is implemented on this asymmetrical relationship.
\par
Up to the point of this writing, there is no known private encryption key from which it is possible to derive a \textit{zero address}, i.e., an account address composed solely of zeroes (0x00000...), which implies that no one "controls" that address. No one has the private key that can be used to sign transactions to move tokens out of this address to another location. But anyone is free to transfer a token that he or she owns to this zero-address. As such, this address has been used historically to "burn" tokens, which consists of transferring a token (NFT or cryptocurrency) to an unrecoverable address. Once a token goes into the \textit{zero address} account, no one can move it out of it due to the lack of a valid private key that can sign the required transaction. Therefore, "burned" tokens are considered as if they were destroyed, when in reality they are stored in an unaccessible account \cite{Antonopoulos2018}.
\par
This approach is interesting and has enough application potential to an e-voting system to warrant a more in-depth exploration. The ability of an NFT to store data directly on the chain while simultaneously maintaining close ownership of the digital object used to modulate this data makes NFTs an interesting candidate for an abstraction of digital voting ballots, while exploring the differences of using a contract-based blockchain versus a account-based approach.

\subsection{Flow Blockchain}
This proposal is based in two implementations in two quite different blockchains. We omit the introduction to the Ethereum blockchain on the assumption that, since it is the most used blockchain in academic research, it is easy for the reader to familiarize himself/herself with it if needed. The Flow blockchain is still relatively unknown, especially within the academic context. This section provides a brief introduction to this blockchain, for context.
\par
The Flow blockchain was formally launched in October 2020 through the usual \textit{Initial Coin Offering (ICO)} that helped launch most public blockchains in recent years. Flow followed a trend that was addressing the high energy consumption associated to earlier, \textit{Proof-of-Work} based blockchains, and established itself with the more efficient and energy friendly \textit{Proof-of-Stake (PoS)} consensus mechanism \cite{Hentschel2019a}.
\par
In a Proof-of-Stake based blockchain, new blocks are "mined" by active nodes based on their \textit{stake} on the network, namely, the volume of the blockchain's native cryptocurrency the node currently holds in its account. Higher stakes mean higher probability of being awarded a block publication by the protocol that regulates the network, with the token incentives that such entitles, and vice versa. This process is more energy efficient and faster than the "classical" PoW consensus, where nodes pointlessly spend computer cycles solving cryptographic puzzles only to win a race towards its completion. Given the significant energy waste due to the popularity of PoW blockchains, such as Bitcoin, a blockchain that stays clear of such nefarious protocol is not just a smart choice but a conscientious one as well \cite{Hentschel2019b}.
\par
Flow was created by the same team that created the \textit{CryptoKitties} project in Ethereum, one of the first NFTs based blockchain games in this network to adopt the \textit{ERC-721} standard for Non-Fungible Tokens. Even though Ethereum was the first one to offer NFT support, it did so on top of a chain that did not envision such applications at the time of its conception. This was evident by the lack of flexibility, inherent security flaws in the code implementing the tokens and its mechanics, as well as a general complexity in writing code and operating with these constructs in the Ethereum network. Ethereum was so ill prepared to deal with the popularity of this initiative that its network went down briefly due to its inability to cope with the added network traffic \cite{bbc2017}. Flow was developed with NFTs support as its main feature, prioritizing NFT creation and mechanics over other applications, a clear contrast over blockchains focused in cryptocurrency transactions \cite{Gharegozlou2019}.
\par
Flow establishes a new computational paradigm in this context, namely a \textit{Resource Oriented Paradigm (ROP)}. It does so through \textit{Cadence}, a smart contract programming language developed for Flow. \textit{Cadence} establishes \textit{ROP} through a special type of object, named \textit{Resource}. From a technological point of view, a Resource is akin to a structure or an object from an Object-Oriented context, offering a significant degree of freedom on the type of data that it can encode. But Cadence regulates operations through the notion that every Resource is unique in Flow. This means that Resources need to be accounted at all times, and they cannot be copied, only moved. They can be stored in decentralized storage accounts and loaded from them but, at all times, the Resource is either in storage or out of storage, never in the same state at the same time. These Resources can be created only through smart contracts functions and are exclusive to the Flow blockchain, i.e., not interchangeable with other NFTs from other blockchains.
\subsubsection{Account-based Storage in Flow}
Flow was conceived to be a NFT-oriented blockchain, unlike the usual, cryptocurrency focus of other alternatives. As such, data storage was emphasized regarding other blockchain characteristics. Other blockchains, Ethereum being a prime example, centralize NFT storage in the respective contract, i.e., the digital object - owning account key-value pair is saved related to the contract. This means that, if the contract is deleted, ownership of its issued NFTs disappears with it, as well as the contract-based NFTs. This mechanic was substantially changed in Flow. Storage in this blockchain is account-based and every NFT created has a specific data type inherited from the contract that minted it, which allows these NFT's to exist outside of the contracts that implement them. Creating an account in Flow is a process similar to creating a cryptocurrency wallet in any other blockchain in the sense that it is also based in the creation of an asymmetric encryption key pair, where the account address is derived from the public key and account access requires control of the corresponding private key. Storage also follows a key-value scheme, as it is common in this context, but every key stored is prefixed by the account address, which is also unique for every user in the Flow ecosystem. A Flow NFT stored in an account is uniquely identified by a concatenation between the account address of the owner of the token, followed by the name of the contract that implements the token and the name of the token itself, using a period to separate the elements. For example, consider a Flow NFT named "ExampleToken" that was defined in a smart contract named "ExampleContract" and it is owned by account 0x1234. This token is uniquely identified in the Flow blockchain by the unique string "0x1234.ExampleContract.ExampleToken". Storage space depends on the amount of Flow token currently deposited in the account, according to a rate of 100MB of storage per token \cite{flow2020b}.
\par
Users control the visibility of the data stored in their accounts by issuing capabilities to other accounts, which can be controlled by other users or other smart contracts. Flow uses different storage domains to implement these visibility levels. By default, once a resource is stored in an account's storage, it goes to the \textbf{/storage} domain where only the owner can access it and modify it. This user can then issue a capability that essentially creates a public reference in the \textbf{/public} domain. The capability is able to filter which parameters are visible to other users and any interactions happen always through the reference, i. e., never the actual digital object \cite{Hentschel2019a}.

\subsection{The case for Flow}
Flow strengthens its case as a prime candidate for a NFT-based project by offering a relatively high block rate and low gas fees, which translates into fast transactions and low overheads. It also uses a novel way to store data, using an account-based storage instead of the "traditional" contract-based storage in Ethereum and similar blockchains. The mechanism of storing resources under the user's account, it is uniquely apt to a context where sensible data needs to be transferred through the blockchain. Securing this data under an individual account, and set it private so that only the account owner can access it, is a feature that distinguishes this blockchain from similar ones. It is easy to understand the security problems behind a more "conventional" blockchain storage system, where all data is referenced from a single point of access, typically the regulating contract address, a centralizing element in a purely decentralized application.
\end{document}